%%%%% Setting the input encoding of the files: UTF-8
%%%%% ---------------------------------------------------------------
\usepackage[utf8]{inputenc}

%%%%% Setting the language of the document
%%%%% ---------------------------------------------------------------
\usepackage{setspace}
\onehalfspacing
\usepackage{indentfirst}
\setlength{\parindent}{0pt}
\setlength{\parskip}{0.75\baselineskip}

%%%%% Color definitions
%%%%% ---------------------------------------------------------------
\usepackage{xcolor}
\definecolor{unicorn_blue}{HTML}{0B2A70}
\definecolor{unicorn_blue_light}{HTML}{40C8D3}
\definecolor{tw_gray}{HTML}{6b7280}

% chart color palette (sync with _chart_utils.py)
\definecolor{chart1}{HTML}{0B2A70}  % Dominant Thesis Color (Dark Blue)
\definecolor{chart2}{HTML}{40C8D3}  % Secondary Thesis Color (Light Blue)
\definecolor{chart3}{HTML}{6C969D}  % Light Teal Gray
\definecolor{chart4}{HTML}{97CC04}  % Vibrant Green
\definecolor{chart5}{HTML}{EEB902}  % Bright Yellow
\definecolor{chart6}{HTML}{F28E2B}  % Warm Orange
\definecolor{chart7}{HTML}{E15759}  % Vibrant Red
\definecolor{chart8}{HTML}{BAB0AC}  % Neutral Gray
\definecolor{chart9}{HTML}{84C1FF}  % Soft Blue
\definecolor{chart10}{HTML}{FF8C94} % Coral Pink

% macro for each color
\newcommand{\getchartcolor}[1]{%
	\ifcase
		#1\relax
		chart1\or % 0 -> chart1
		chart1\or % 0 -> chart1
		chart2\or % 1 -> chart2
		chart3\or % 2 -> chart3
		chart4\or % 3 -> chart4
		chart5\or % 4 -> chart5
		chart6\or % 5 -> chart6
		chart7\or % 6 -> chart7
		chart8\or % 7 -> chart8
		chart9\or % 8 -> chart9
		chart10%   9 -> chart10
	\fi
}


%%%%% Setting the color boxes/frames
%%%%% ---------------------------------------------------------------
\usepackage{tcolorbox}
\usepackage{fontawesome5}
% color box presets
\newtcolorbox{gray-box}[1]{colback=gray!5!white,colframe=gray!50!black,title=#1}
\newtcolorbox{blue-box}[1]{colback=unicorn_blue!5!white,colframe=unicorn_blue,title=#1}
\newtcolorbox{light-gray-box}[1]{colback=tw_gray!2,colframe=tw_gray!10,title={\textcolor{black}{\faQuestionCircle[regular]~~#1}}}

% research question box as environment
\NewDocumentEnvironment{rqbox}{m}{
	\begin{gray-box}{Research Question}
		\small
		\rqfull{#1}
		}{
	\end{gray-box}
}
\newcommand{\makerqbox}[1]{\begin{rqbox}{#1}\end{rqbox}}

% key takeaways box as environment
\NewDocumentEnvironment{keytakeaways}{}{
	\begin{blue-box}{Key Takeaways}
		\begin{flushleft}
			\small
			}{
		\end{flushleft}
	\end{blue-box}
}

% info box as environment
\NewDocumentEnvironment{infobox}{m}{
	\begin{light-gray-box}{#1}
		\begin{flushleft}
			\small
			}{
		\end{flushleft}
	\end{light-gray-box}
}


%%%%% Setting the colors and syntax of the code
%%%%% ---------------------------------------------------------------
% TODO: Define the colors
\usepackage{fvextra}
\usepackage{colortbl} % for colored rows/columns
\definecolor{codekeyword}{rgb}{0.0,0.3,0.7}
\definecolor{codestring}{rgb}{0.4,0.4,0.8}
\definecolor{codecomment}{rgb}{0.2,0.2,0.2}
\definecolor{codejsdoc}{rgb}{0.4,0.4,0.4}
\definecolor{codelinenum}{rgb}{0.8,0.8,0.8}
\definecolor{lightyellow}{rgb}{255, 255, 224}

% minted package setup
\usepackage[outputdir=dist]{minted}
\setminted{
	frame=none,
	breaklines=true,
	fontsize=\footnotesize,
	tabsize=2,
	linenos,
	numbersep=5pt,
	xleftmargin=0pt,
	baselinestretch=1.2,
	style=friendly,
%   TODO: highlighting not working
	highlightcolor=\color{lightyellow},
	keywordstyle=\color{codekeyword},
	stringstyle=\color{codestring},
	commentstyle=\color{codecomment}\itshape,
	morecomment=[s][\color{codejsdoc}]{/**}{*/},
	numberstyle=\footnotesize\color{codelinenum}
}

% listings package setup
\usepackage{listings}
\lstset{
	basicstyle=\ttfamily,
	columns=fullflexible,
	frame=single,
	breaklines=true,
	showstringspaces=false,
	keywordstyle=\bfseries,
	commentstyle=\itshape\color{gray},
	captionpos=t
}
% make listings follow chapter numbering (e.g., 1.1, 1.2, ...)
\usepackage{chngcntr}
\counterwithin{listing}{chapter}
\renewcommand{\thelisting}{\thechapter.\arabic{listing}}

% wrap long URLs
\PassOptionsToPackage{hyphens}{url}

%%%%% Tables
%%%%% ---------------------------------------------------------------
\usepackage{tabularx}
\usepackage{csvsimple}
\usepackage{booktabs}

%%%%% Chart environment setup
%%%%% ---------------------------------------------------------------
\usepackage{svg}
\usepackage{newfloat}
\DeclareFloatingEnvironment[
	fileext=loc,
	listname={List of Charts},
	name=Chart,
	placement=tbhp,
	within=chapter
]{chart}

% Setup autoref for charts
\makeatletter
\newcommand*{\chartautorefname}{\textbf{Chart}}
\makeatother

%%%%% Additional useful packages
%%%%% ----------------------------------------------------------------
\usepackage{amsmath}
\usepackage{amsfonts}
\usepackage{amsthm}
\usepackage{bm}
\usepackage{graphicx}
\usepackage[labelfont=bf]{caption}
\newcommand{\sourceDefaultLabel}{Author's rendition}
\newcommand{\sourceLabel}{Source:}
\newcommand{\source}[1][\sourceDefaultLabel]{\caption*{\hfill\footnotesize{\sourceLabel~\textit{#1}}}}
\usepackage{psfrag}
\usepackage{fancyvrb}
\usepackage[numbers]{natbib}
\usepackage{usebib}
\usepackage{tikz}
\usepackage{bbding}
\usepackage{icomma}
\usepackage{dcolumn}
\usepackage{paralist}
\usepackage{float}
\usepackage{subcaption}
\usepackage{epigraph}
\newcommand\foreign[1]{\emph{#1}}
\usepackage{pdfpages}
\usepackage{nameref}
% will create a reference to a chapter, section, ... including the number and in bold
\renewcommand{\fullref}[1]{\textbf{\nameref{#1}}}

% utils
\newcommand{\todo}[1]{\newline\textcolor{red}{\textbf{TODO:} #1}\newline}
\newcommand{\note}[1]{\newline\textcolor{blue}{\textbf{NOTE:} #1}\newline}
\newcommand{\theEvent}{\textbf{The Event}}
\newcommand{\theOrganizer}{\textbf{The Organizer}}
\newcommand{\eg}{e.g.,}

% Formatting numbers
\usepackage{siunitx} % for number formatting
% configure siunitx for Czech number formatting
\sisetup{
	group-separator={~},
	group-minimum-digits=4,
	round-mode=places,
	round-precision=0,
	output-decimal-marker={.},
}
\newcommand{\fmtnum}[1]{\num{#1}}
\newcommand{\fmtnump}[2][2]{\num[round-mode=places, round-precision=#1]{#2}}
\newcommand{\fmtczk}[1]{\num{#1}~CZK}
\newcommand{\fmtczkp}[2][2]{\num[round-mode=places, round-precision=#1]{#2}~CZK}
% bold (using mathbf + textbf)
\newcommand{\bfmtnum}[1]{\boldmath\textbf{\num[reset-math-version=false]{#1}}}
\newcommand{\bfmtnump}[2][2]{\boldmath\textbf{\num[reset-math-version=false, round-mode=places, round-precision=#1]{#2}}}
\newcommand{\bfmtczk}[1]{\boldmath\textbf{\num[reset-math-version=false]{#1}~CZK}}
\newcommand{\bfmtczkp}[2][2]{\boldmath\textbf{\num[reset-math-version=false, round-mode=places, round-precision=#1]{#2}~CZK}}

% Command for color box indicator
\newcommand{\colorindicatorhex}[1]{%
	\textcolor[HTML]{#1}{\rule{0.8em}{0.8em}}\hspace{0.5em}%
}
\newcommand{\colorindicator}[1]{%
	\textcolor{\getchartcolor{#1}}{\rule{0.8em}{0.8em}}\hspace{0.5em}%
}

%%%%% Setting the acronyms
%%%%% ---------------------------------------------------------------
\usepackage{acro}
\DeclareAcronym{api}{short=API,     long=Application Programming Interface }

%%%%% Setting the headings
%%%%% ------------------------------------------------------------
\usepackage{titlesec}
%\titlespacing*{\chapter}{0pt}{-10mm}{5mm}
\titleformat{\chapter}{\normalfont\huge\bfseries}{\thechapter}{1em}{}

%%%%% Counters and questions
%%%%% ------------------------------------------------------------
\usepackage{totcount}
\usepackage{xparse}
% counter for questions
\newcounter{rqcounter}
\regtotcounter{rqcounter}

% Configure autoref format for research questions
\makeatletter
% @formatter:off
\def\therqcounter{\ifnum\value{rqcounter}<10\relax0\fi\number\value{rqcounter}}
% @formatter:on
\@addtoreset{rqcounter}{chapter}
\def\autoref@rqcounter#1{\RQ{#1}}
\def\RQ#1{\textup{RQ#1}}
\makeatother

% Command to define a new research question
\NewDocumentCommand{\defresearchq}{m m}{
	\refstepcounter{rqcounter}
	% save RQ number
	\expandafter\edef\csname rqnum:\string#1\endcsname{\therqcounter}
	% save RQ label (RQ#)
	\expandafter\edef\csname rqlabel:\string#1\endcsname{RQ\therqcounter}
	% define label
	\expandafter\label\expandafter{rq:#1}
	% save question text
	\expandafter\edef\csname rqtext:\string#1\endcsname{#2}
	% full formatted RQ
	\expandafter\newcommand\csname rq:\string#1\endcsname{
		\textbf{\csname rqnum:\string#1\endcsname:} #2
	}
}


% use a research question raw text
\NewDocumentCommand{\rqtext}{m}{
	\csname rqtext:\string#1\endcsname
}

% use a research question number (# only)
\NewDocumentCommand{\rqnumber}{m}{
	\csname rqnum:\string#1\endcsname
}

% use a research question (RQ#: Question text?)
\NewDocumentCommand{\rqfull}{m}{%
	\hyperref[rq:#1]{%
		\textbf{\csname rqlabel:\string#1\endcsname}\nobreak: \rqtext{#1}%
	}%
}

% use a research question shortcut (RQ#)
\NewDocumentCommand{\rqshort}{m}{
	\hyperref[rq:#1]{\textbf{\csname rqlabel:\string#1\endcsname}}
}

% use a research question shortcut (RQ#) - non-hyperlinked
\NewDocumentCommand{\rqshorttext}{m}{
	\csname rqlabel:\string#1\endcsname\textbf{}
}

% @formatter:off
\renewcommand{\therqcounter}{\ifnum\value{rqcounter}<10\relax0\fi\number\value{rqcounter}}
% @formatter:on
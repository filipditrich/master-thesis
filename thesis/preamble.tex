%%%%% Setting the input encoding of the files: UTF-8
%%%%% ---------------------------------------------------------------
\usepackage[utf8]{inputenc}

%%%%% Setting the language of the document
%%%%% ---------------------------------------------------------------
\usepackage{setspace}
\onehalfspacing
\usepackage{indentfirst}
\setlength{\parindent}{0pt}
\setlength{\parskip}{0.75\baselineskip}

%%%%% Color definitions
%%%%% ---------------------------------------------------------------
\usepackage{xcolor}
\definecolor{unicorn_blue}{HTML}{0B2A70}
\definecolor{unicorn_blue_light}{HTML}{40C8D3}
\definecolor{chart1}{HTML}{2D7DD2}  % Clear Blue
\definecolor{chart2}{HTML}{6C969D}  % Golden Yellow
\definecolor{chart3}{HTML}{97CC04}  % Lime Green
\definecolor{chart4}{HTML}{EEB902}  % Orange
\definecolor{chart5}{HTML}{474647}  % Dark Gray
\definecolor{chart6}{HTML}{F45D01}  % Steel Blue
\definecolor{chart7}{HTML}{9B6B6C}  % Dusty Rose
\definecolor{chart8}{HTML}{556F44}  % Forest Green

%%%%% Setting the color boxes/frames
%%%%% ---------------------------------------------------------------
\usepackage{tcolorbox}
% gray box preset
\newtcolorbox{gray-box}[1]{colback=gray!5!white,colframe=gray!50!black,title=#1}
\newtcolorbox{blue-box}[1]{colback=unicorn_blue!5!white,colframe=unicorn_blue,title=#1}


%%%%% Setting the colors and syntax of the code
%%%%% ---------------------------------------------------------------
% TODO: Define the colors
\usepackage{fvextra}
\usepackage{colortbl} % for colored rows/columns
\definecolor{codekeyword}{rgb}{0.0,0.3,0.7}
\definecolor{codestring}{rgb}{0.4,0.4,0.8}
\definecolor{codecomment}{rgb}{0.2,0.2,0.2}
\definecolor{codejsdoc}{rgb}{0.4,0.4,0.4}
\definecolor{codelinenum}{rgb}{0.8,0.8,0.8}
\definecolor{lightyellow}{rgb}{255, 255, 224}

% minted package setup
\usepackage[outputdir=dist]{minted}
\setminted{
	frame=none,
	breaklines=true,
	fontsize=\footnotesize,
	tabsize=2,
	linenos,
	numbersep=5pt,
	xleftmargin=0pt,
	baselinestretch=1.2,
	style=friendly,
%   TODO: highlighting not working
	highlightcolor=\color{lightyellow},
	keywordstyle=\color{codekeyword},
	stringstyle=\color{codestring},
	commentstyle=\color{codecomment}\itshape,
	morecomment=[s][\color{codejsdoc}]{/**}{*/},
	numberstyle=\footnotesize\color{codelinenum}
}

% listings package setup
\usepackage{listings}
\lstset{
	basicstyle=\ttfamily,
	columns=fullflexible,
	frame=single,
	breaklines=true,
	showstringspaces=false,
	keywordstyle=\bfseries,
	commentstyle=\itshape\color{gray},
	captionpos=t
}

% wrap long URLs
\PassOptionsToPackage{hyphens}{url}

%%%%% Tables
%%%%% ---------------------------------------------------------------
\usepackage{tabularx}

%%%%% Additional useful packages
%%%%% ----------------------------------------------------------------
\usepackage{amsmath}
\usepackage{amsfonts}
\usepackage{amsthm}
\usepackage{bm}
\usepackage{graphicx}
\usepackage[labelfont=bf]{caption}
\newcommand{\sourceDefaultLabel}{Own creation}
\newcommand{\sourceLabel}{Source:}
\newcommand{\source}[1][\sourceDefaultLabel]{\caption*{\hfill\footnotesize{\sourceLabel~\textit{#1}}}}
\usepackage{psfrag}
\usepackage{fancyvrb}
\usepackage[numbers]{natbib}
\usepackage{usebib}
\usepackage{tikz}
\usepackage{bbding}
\usepackage{icomma}
\usepackage{dcolumn}
\usepackage{booktabs}
\usepackage{paralist}
\usepackage{float}
\usepackage{subcaption}
\usepackage{epigraph}
\newcommand\foreign[1]{\emph{#1}}
\usepackage{pdfpages}
\usepackage{nameref}
% will create a reference to a chapter, section, ... including the number and in bold
\renewcommand{\fullref}[1]{\textbf{\nameref{#1}}}

% utils
\newcommand{\todo}[1]{\newline\textcolor{red}{\textbf{TODO:} #1}\newline}
\newcommand{\note}[1]{\newline\textcolor{blue}{\textbf{NOTE:} #1}\newline}
\newcommand{\theEvent}{\textbf{The Event}}
\newcommand{\theOrganizer}{\textbf{The Organizer}}
\newcommand{\eg}{e.g.,}

% Formatting numbers
\usepackage{siunitx} % for number formatting
% configure siunitx for Czech number formatting
\sisetup{
	group-separator={,},
	group-minimum-digits=4,
	round-mode=places,
	round-precision=0,
	output-decimal-marker={.},
}
\newcommand{\fmtnum}[1]{\num{#1}}
\newcommand{\fmtnump}[2][2]{\num[round-mode=places, round-precision=#1]{#2}}
\newcommand{\fmtczk}[1]{\num{#1}~CZK}
\newcommand{\fmtczkp}[2][2]{\num[round-mode=places, round-precision=#1]{#2}~CZK}
% bold (using mathbf + textbf)
\newcommand{\bfmtnum}[1]{\boldmath\textbf{\num[reset-math-version=false]{#1}}}
\newcommand{\bfmtnump}[2][2]{\boldmath\textbf{\num[reset-math-version=false, round-mode=places, round-precision=#1]{#2}}}
\newcommand{\bfmtczk}[1]{\boldmath\textbf{\num[reset-math-version=false]{#1}~CZK}}
\newcommand{\bfmtczkp}[2][2]{\boldmath\textbf{\num[reset-math-version=false, round-mode=places, round-precision=#1]{#2}~CZK}}

% Command for color box indicator
\newcommand{\colorindicatorhex}[1]{%
	\textcolor[HTML]{#1}{\rule{0.8em}{0.8em}}\hspace{0.5em}%
}
\newcommand{\colorindicator}[1]{%
	\textcolor{#1}{\rule{0.8em}{0.8em}}\hspace{0.5em}%
}

%%%%% Setting the acronyms
%%%%% ---------------------------------------------------------------
\usepackage{acro}
\DeclareAcronym{api}{short=API,     long=Application Programming Interface }

%%%%% Setting the headings
%%%%% ------------------------------------------------------------
\usepackage{titlesec}
%\titlespacing*{\chapter}{0pt}{-10mm}{5mm}
\titleformat{\chapter}{\normalfont\huge\bfseries}{\thechapter}{1em}{}

%%%%% Counters and questions
%%%%% ------------------------------------------------------------
\usepackage{totcount}
\usepackage{xparse}
% counter for questions
\newcounter{rqcounter}
\regtotcounter{rqcounter}

% Configure autoref format for research questions
\makeatletter
\def\therqcounter{\@arabic\c@rqcounter}
\@addtoreset{rqcounter}{chapter}
\def\autoref@rqcounter#1{\RQ{#1}}
\def\RQ#1{\textup{RQ#1}}
\makeatother

% Command to define a new research question
\NewDocumentCommand{\defresearchq}{m m}{
	\refstepcounter{rqcounter}
	\expandafter\edef\csname rqnum:\string#1\endcsname{\therqcounter}
	\expandafter\label\expandafter{rq:#1}
	\expandafter\newcommand\csname rq:\string#1\endcsname{
		\textbf{\csname rqnum:\string#1\endcsname:} #2
	}
}

% Command to use a research question
\NewDocumentCommand{\researchq}{m}{
	\hyperref[rq:#1]{\csname rq:\string#1\endcsname}
}

\renewcommand{\therqcounter}{RQ\arabic{rqcounter}}
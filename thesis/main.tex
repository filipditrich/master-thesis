%%%%% Basic setup for one-sided printing:
%%%%% Margins: left 40mm, right 25mm, top and bottom 25mm (but beware, LaTeX adds 1in by itself)
%%%%% ---------------------------------------------------------------
\documentclass[12pt, a4paper]{report}
\usepackage{geometry}
\setlength\textwidth{145mm}
\setlength\textheight{247mm}
\setlength\oddsidemargin{15mm}
\setlength\evensidemargin{15mm}
\setlength\topmargin{0mm}
\setlength\headsep{0mm}
\setlength\headheight{0mm}
\newcommand{\openright}{\clearpage}

%%%%% Base setup for two-sided printing:
%%%%% ---------------------------------------------------------------
% \documentclass[12pt, a4paper, twoside, openright]{report}
% \setlength\textwidth{145mm}
% \setlength\textheight{247mm}
% \setlength\oddsidemargin{15mm}
% \setlength\evensidemargin{0mm}
% \setlength\topmargin{0mm}
% \setlength\headsep{0mm}
% \setlength\headheight{0mm}
% \let\openright=\cleardoublepage

%%%%% Czech language settings
%%%%% ---------------------------------------------------------------
% \usepackage[czech]{babel}
% \ifx\uv\undefined\newcommand{\uv}[1]{,,#1``}\fi
\usepackage [english]{babel}
\usepackage [autostyle, english = american]{csquotes}
\MakeOuterQuote{"}

%%%%% Preamble with settings and macros
%%%%% ---------------------------------------------------------------
%%%%% Setting the input encoding of the files: UTF-8
%%%%% ---------------------------------------------------------------
\usepackage[utf8]{inputenc}

%%%%% Setting the language of the document
%%%%% ---------------------------------------------------------------
\usepackage{setspace}
\onehalfspacing
\usepackage{indentfirst}
\setlength{\parindent}{0pt}
\setlength{\parskip}{0.75\baselineskip}

%%%%% Setting the color boxes/frames
%%%%% ---------------------------------------------------------------
\usepackage{tcolorbox}
% gray box preset
\newtcolorbox{gray-box}[1]{colback=gray!5!white,colframe=gray!50!black,title=#1}
\definecolor{unicorn_blue}{HTML}{0B2A70}
\newtcolorbox{blue-box}[1]{colback=unicorn_blue!5!white,colframe=unicorn_blue,title=#1}


%%%%% Setting the colors and syntax of the code
%%%%% ---------------------------------------------------------------
% TODO: Define the colors
\usepackage{fvextra}
\usepackage{xcolor}
\usepackage{colortbl} % for colored rows/columns
\definecolor{codekeyword}{rgb}{0.0,0.3,0.7}
\definecolor{codestring}{rgb}{0.4,0.4,0.8}
\definecolor{codecomment}{rgb}{0.2,0.2,0.2}
\definecolor{codejsdoc}{rgb}{0.4,0.4,0.4}
\definecolor{codelinenum}{rgb}{0.8,0.8,0.8}
\definecolor{lightyellow}{rgb}{255, 255, 224}

% minted package setup
\usepackage[outputdir=dist]{minted}
\setminted{
	frame=none,
	breaklines=true,
	fontsize=\footnotesize,
	tabsize=2,
	linenos,
	numbersep=5pt,
	xleftmargin=0pt,
	baselinestretch=1.2,
	style=friendly,
%   TODO: highlighting not working
	highlightcolor=\color{lightyellow},
	keywordstyle=\color{codekeyword},
	stringstyle=\color{codestring},
	commentstyle=\color{codecomment}\itshape,
	morecomment=[s][\color{codejsdoc}]{/**}{*/},
	numberstyle=\footnotesize\color{codelinenum}
}

% listings package setup
\usepackage{listings}
\lstset{
	basicstyle=\ttfamily,
	columns=fullflexible,
	frame=single,
	breaklines=true,
	showstringspaces=false,
	keywordstyle=\bfseries,
	commentstyle=\itshape\color{gray},
	captionpos=t
}

% wrap long URLs
\PassOptionsToPackage{hyphens}{url}

%%%%% Additional useful packages
%%%%% ----------------------------------------------------------------
\usepackage{amsmath}
\usepackage{amsfonts}
\usepackage{amsthm}
\usepackage{bm}
\usepackage{graphicx}
\usepackage[labelfont=bf]{caption}
\newcommand{\sourceDefaultLabel}{Own creation}
\newcommand{\sourceLabel}{Source:}
\newcommand{\source}[1][\sourceDefaultLabel]{\caption*{\hfill\footnotesize{\sourceLabel~\textit{#1}}}}
\usepackage{psfrag}
\usepackage{fancyvrb}
\usepackage[numbers]{natbib}
\usepackage{usebib}
\usepackage{tikz}
\usepackage{bbding}
\usepackage{icomma}
\usepackage{dcolumn}
\usepackage{booktabs}
\usepackage{paralist}
\usepackage{float}
\usepackage{subcaption}
\usepackage{epigraph}
\newcommand\foreign[1]{\emph{#1}}
\usepackage{pdfpages}
\usepackage{nameref}
% will create a reference to a chapter, section, ... including the number and in bold
\renewcommand{\fullref}[1]{\textbf{\nameref{#1}}}

% utils
\newcommand{\todo}[1]{\newline\textcolor{red}{\textbf{TODO:} #1}\newline}
\newcommand{\note}[1]{\newline\textcolor{blue}{\textbf{NOTE:} #1}\newline}
\newcommand{\theEvent}{\textbf{The Event}}
\newcommand{\theOrganizer}{\textbf{The Organizer}}
\newcommand{\eg}{e.g.,}


%%%%% Setting the acronyms
%%%%% ---------------------------------------------------------------
\usepackage{acro}
\DeclareAcronym{api}{short=API,     long=Application Programming Interface }

%%%%% Setting the headings
%%%%% ------------------------------------------------------------
\usepackage{titlesec}
%\titlespacing*{\chapter}{0pt}{-10mm}{5mm}
\titleformat{\chapter}{\normalfont\huge\bfseries}{\thechapter}{1em}{}


%%%%% Hyperlinks setup
%%%%% ------------------------------------------------------------
\usepackage[unicode]{hyperref}
\hypersetup{pdftitle=Cashless festival data analysis and analytical dashboard development,
	pdfauthor=Bc. Filip Ditrich
	ps2pdf,
	colorlinks=true,
	urlcolor=black,
	linkcolor=black,
	citecolor=black,
	pdfstartview=FitH,
	pdfpagemode=UseOutlines,
	pdfnewwindow,
	breaklinks
}

%%%%% Bibliography setup
%%%%% ---------------------------------------------------------------
\bibliographystyle{/Users/filipditrich/University/master_thesis/thesis/czplainnat.bst}
\renewcommand{\bibname}{References}

%%%%% Table of contents setup
%%%%% ---------------------------------------------------------------
\usepackage[nottoc]{tocbibind}
\renewcommand{\listingscaption}{Source code}
\renewcommand{\listoflistingscaption}{List of Source Codes}
\usepackage[titles]{tocloft}
\renewcommand\cftchapafterpnum{\vskip0pt}
\renewcommand\cftsecafterpnum{\vskip2pt}
\usepackage[toc,page]{appendix}
% \renewcommand{\appendixtocname}{Seznam příloh}
% \renewcommand{\appendixpagename}{Seznam příloh}

%%%%% Directories setup
%%%%% ---------------------------------------------------------------
\newcommand{\ThesisFigures}{./figures}
\newcommand{\ResultsDir}{./results}
\newcommand{\CoreFigures}{../figures}

%%%%% TODO:
%% - [x] setup the structure and technical setup
%% - [] add copy of the assignment PDF

% TODO: Text revision notes
%% - [x] NOTE: In cooperation with the event organizer: I have chose a "not more undisclosed event organizer" (str 14) - je to správně? Nerozumím tomu.
%% - [x] NOTE: The Event definition: The Event takes place in the begging (str 15)
%% - [x] NOTE: Cashflow and revenue: "these questions should provide currently unclear insights" - podivná formulace, nerozumím jí
%% - [x] NOTE: RQ24: "how much only once" - nemá to být "how many"?
%% - [x] NOTE: Limitations: Anonymized data: "...which may limit certain the analysis in some ways" tu něco nesedí (str 20)
%% - [x] NOTE: Data and methodology: "This chapter address" => addresses
%% - [x] NOTE: Best Sale points: str. 50, přeposlední odstavec: "Another interesting fact is the total unique users processed at the places" - tomu nerozumím - nechybí v té větě neco?
%% - [] NOTE: "Out of total of 145 sale points, the best place was undeniable the L20 PIVNÍ STAN 1" - tady jsem se rozesmál. Je to důležité, ale zároveň naprosto samozřejmé zjištění.

% TODO: Graphical revision notes (charts)
%% - [x] NOTE: Table 1.1: překlep - coffe
%% - [] NOTE: Každý blok research question zakončujete frází typu: "answers to these question should provide valuable/detailed insights" občas to obměňte
%% - [] NOTE: Na několika místech začínáte novou větu (nikoliv otázku) slovem "Which" (např. 2. odstavec sekce 2.2, 3. odstavec v introduction a možná ještě někde). Mám dojem, že to angličtině nedovoluje - which je vztažné zájmenou a mělo by podle mě uvozovat vedlejší větu v souvětí. Možná se mýlím, ale připadá mi to divné.
%% - [] NOTE: Figure 1.1: nevím, jestli je to jen u mě, ale to schéma se mi renderuje v mizerné kvalitě na hranici čitelnosti
%% - [] NOTE: Figure 2.1: Zvažte, jestli chcete tento typ grafu. Na donut chart to má trochu malou díru, až to hraničí z pie chartem. Pie charts jsou lidově populární, ale datasci lidé je nemají rádi (z dobrých důvodů). Jako alternativy se nabízí např. horizontální barplot, treemap nebo waflle (aspoň myslím) diagram. Tohle je low-priority poznámka. Řešil bych to, až když nebudete mít co jiného dělat. Pokud práci bude oponovat např. Karel Šafr, pravděpodobně se u těchto grafů zastaví ;)
%% - [] NOTE: Figure 2.1: Necháte-li donut/pie chart, alespoň bych doplnil ta procenta do přidružené tabulky.
%% - [] NOTE: Str. 37: o těch overfull boxes kvůli 3,471,543 CZK a 684,700 CZK asi víte.
%% - [] NOTE: Figure 2.2: podle mě mírně nešťastný. Ty grafy vedle sebe se trochu špatně porovnávají. Navíc každý znich má jinou yscale, tj. výška sloupce dobře nepřenáší hodnotu. Organizer beer vypadá na první pohled stejně jako external salty. Sloučil bych je do jednoho, nejspíš jako horizontální barplot. Každý bar by odpovídal jedné kategorii produktu, a měl by dvě části (třeba světlou a tmavou dané barvy), jednu reprezentující podíl organizor vendors na prodeji proudktu, druhou podíl external vendors. Tahle varianta více zdůrazňuje jejich rozdílné zastoupení v katergoriích. Alternativně, pokud byste chtěl zdůraznit spíše produktové složení sales organizer vs. external, představoval bych si dva horizontální bars (organizer a external), každý rozdělený na sekce podle produktů. V této druhé variantě bude navíc lépe vidět podíl organizer vs external na celkových sales. Záleží na tom, co chcete spíše zdůraznit. Navíc myslím, že non-alcoholic a beer jste označil stejnou barvou.
%% - [] NOTE: Figure 2.3: souhlasím. Sankey diagram je cool, ale tady to zabíjí dominantní spent on sales větev. Něco jako waffle chart?
%% - [] NOTE: Figure 2.4: tady imho ten donut/pie chart vadí ze všeho nejméně, protože ta data jsou jednoduchá a dobře čitelná. Ale zvolil bych ho nakonec nějak konzistentně s předchozími grafy.
%% - [] NOTE: Figure 2.5: asi stejně jako Figure 2.1
%% - [] NOTE: Figure 2.6: super.
%% - [] NOTE: Figure 2.7: velmi malý font, špatně čitelné.
%% - [] NOTE: Figure 2.8: to je hádám velmi WIP
%% - [] NOTE: Figure 2.13 a 2.14: mále písmo, špatně čitelné.

%%%%% Main document part
%%%%% ------------------------------------------------------------
\begin{document}
%%% title page
	%%% Hard title page
%%%%%%% Wording: ⏳
%%%%%%% Styling: ⏳
%%%%%%% References: ⏳
%%%%% Grammar: ⏳
%%% --------------------------------------------------------------
\pagestyle{empty}
\begin{center}
{\Large\bfseries\MakeUppercase{Unicorn Vysoká škola s.r.o.}}
    \vfill

    {\Huge\bfseries\MakeUppercase{Master's Thesis}} \\

    \vfill

    \noindent\begin{minipage}{\textwidth}
                 \begin{Large}
                     \textbf{2024} \hfill \textbf{Bc. Filip \MakeUppercase{Ditrich}}
                 \end{Large}
    \end{minipage}
\end{center}
 % FIXME: remove
	%%% Cover page
%%%%%%% Wording: ⏳ (TODO: check for year 24 or 25)
%%%%%%% Styling: ✅
%%%%%%% References: ✅
%%%%% Grammar: ✅
%%% --------------------------------------------------------------
\pagestyle{empty}
\begin{center}

%%% school name
{\bfseries\large UNICORN VYSOKÁ ŠKOLA s.r.o.}

	\vspace{5mm}

	%%% název oboru
	{\Large Software Engineering and Big Data}

	\vfill
	\vspace{5mm}

	%%% logo
	\centerline{\mbox{\includegraphics[width=83.3mm]{\CoreFigures/uu-icon}}}

	\vfill
	\vspace{5mm}

	%%% thesis type
	{\large\MakeUppercase{Diploma Thesis}}

	\vspace{15mm}

	%%% name of the thesis
	{\LARGE\bfseries Cashless festival data analysis and analytical dashboard development}

	\vfill

	%%% author and supervisor
	\begin{tabular}{rl}
		Author:     & Bc.~Filip Ditrich       \\
		\noalign{\vspace{2mm}}
		Supervisor: & Mgr.~Václav Alt,~Ph.~D. \\
	\end{tabular}

	\vfill

	%%% rok
	Prague 2024

\end{center}

%%% TODO: copy of the assignment
%    \includepdf[pages={1}]{\ThesisFigures/zadani-zp.pdf}
%    \includepdf[pages={2}]{\ThesisFigures/zadani-zp.pdf}
%%% declaration
	%%% Statutory Declaration
%%%%%%% Wording: ⏳
%%%%%%% Styling: ⏳
%%%%%%% References: ⏳
%%%%% Grammar: ⏳
%%% --------------------------------------------------------------
\newpage
\pagestyle{empty}
\vspace*{\stretch{8}}

%%% title
\noindent
{\large\bfseries Statutory Declaration}\\

%%% text
\noindent
I hereby declare that I have written my bachelor thesis on the topic of \textit{Cashless festival data analysis and analytical dashboard development} by myself, under the guidance of my thesis supervisor, using only the technical publications and other information sources which are all quoted in the thesis and listed in the bibliography.\\
% I declare that artificial intelligence tools have been used only for support activities and in accordance with the principle of academic ethics.

\noindent
As the author of this bachelor thesis, I also declare that in association with its writing I have not violated the copyright of any third party or parties and I am fully aware of the consequences of provisions of s. 11 et seq. of Act No. 121/2000 Coll., the Copyright Act.\\

\noindent
Furthermore, I hereby declare that the submitted hard copy of this bachelor thesis is identical to the electronic version I have submitted.\\

%%% signature - place/date
\vspace{18mm}
\noindent
In \makebox[4cm]{\dotfill} on \makebox[2.5cm]{\dotfill}
\hspace*{\fill}
\makebox[4cm]{\dotfill}

%%% signature
\begin{flushright}
    \noindent
    Filip Ditrich
\end{flushright}

%%% acknowledgements
	%%% Acknowledgements
%%%%%%% Wording: ⏳
%%%%%%% Styling: ⏳
%%%%%%% References: ⏳
%%%%% Grammar: ⏳
%%% --------------------------------------------------------------
\newpage
\pagestyle{empty}
\vspace*{\stretch{8}}

%%% title
\noindent
{\large\bfseries Acknowledgements}\\

%%% text
\noindent
I would like to express my deepest gratitude to my supervisor, Mgr.~Václav Alt,~Ph.~D., for his exceptional guidance, prompt feedback,
and unwavering moral support throughout this challenging journey, particularly during the most demanding phases of the thesis.

Special thanks go to our partner event organizer and NFCtron for providing the opportunity to conduct this research and for their invaluable assistance during the initial stages of the project.

I am also profoundly grateful to my partner for her endless patience and encouragement, and to my family and friends whose understanding and support made this work possible.
%%% first page
	%%% First page
%%%%%%% Wording: ✅
%%%%%%% Styling: ✅
%%%%%%% References: ✅
%%%%% Grammar: ✅
%%% --------------------------------------------------------------
\newpage
\pagestyle{empty}

%%% edges of the page
\newgeometry{textwidth=100mm, textheight=247mm, left=40.4mm, right=75mm}

%%% background
\tikz[remember picture,overlay]
\node[opacity=1,inner sep=0pt] at (current page.center)
	{\includegraphics[width=\paperwidth,height=\paperheight]{\CoreFigures/side-banner}};

\begin{center}

	%%% logo
	\centerline{\mbox{\includegraphics[width=39.6mm]{\CoreFigures/uu-icon}}}

	\vfill

	%%% thesis name in english
	\Large\textbf{Cashless festival data analysis and analytical dashboard development}

	\vspace{5mm}

	\vfill

	%%% logo
	\centerline{\mbox{\includegraphics[width=45mm]{\CoreFigures/uu-logo}}}

\end{center}
\restoregeometry

%%% abstract
	%%% Abstract
%%%%%%% Wording: ⏳
%%%%%%% Styling: ⏳
%%%%%%% References: ⏳
%%%%% Grammar: ⏳
%%% --------------------------------------------------------------
\newpage
\pagestyle{plain}

%%% abstract in english
\nobreak\vbox to 0.49\vsize{
    \setlength\parindent{0mm}
    \setlength\parskip{5mm}

    %%% title
    {\large\bfseries Abstract}

    %%% text
    \noindent
    TODO

    \textit{Keywords: TODO}
    \vss}

%%% abstract in czech
\vbox to 0.5\vsize{
    \setlength\parindent{0mm}
    \setlength\parskip{5mm}

    %%% title
    {\large\bfseries Abstrakt}

    %%% text
    \noindent
    TODO

    \textit{Klíčová slova: TODO}
    \vss}

%%% table of contents
	\newpage
	\pagestyle{plain}
	\tableofcontents
%%% chapters
	% chapter - introduction
	%%% Introduction
%%%%%%% Wording: ✅
%%%%%%% Styling: ✅
%%%%%%% References: ✅
%%%%% Grammar: ✅
%%% --------------------------------------------------------------
\chapter*{Introduction}
\addcontentsline{toc}{chapter}{Introduction}
\label{ch:introduction}


\section*{Background and Motivation}
\addcontentsline{toc}{section}{Background and Motivation}
\label{sec:introduction-background-motivation}
Payments at festivals are a crucial part of the successful event management.
The shift from cash to cashless payments has been a significant trend in the last decade that has brought many benefits to both festival organizers and attendees\cite{bl_en_waarom_festivals_overstappen_op_cashless_betalen}.

However, traditional cashless payment systems using payment terminals are not only expensive to reliably implement at a venue, where the internet connection is often unreliable, but also do not provide any insights into the data generated by the transactions.
In the best case scenario, the organizers are able to generate a report of processed transactions made by each terminal.

Which, frankly, is not enough to make any actionable decisions based on the data.
Moreover, given the event organizers are not data scientists, they often lack the knowledge and tools to analyze the data and extract valuable insights from it.

That is where NFCtron comes in and offers a solution that not only provides a reliable cashless payment system, with credit-based NFC chip bracelets supporting offline mode or card terminal payment solutions, but also provides a comprehensive B2B platform.
This platform allows organizers, vendors, and event third-party partners to benefit from the data that the system operates with\cite{nfctron_en_company}.

The system is a full-scope solution that provides from initial online ticket sales and online credit top-up, through attendee check-in, on-site credit top-up, attendee access control, security monitoring, vendor sales, and inventory management.
Most importantly, it provides fast and reliable payment processing with the real-time data analytics and reporting, all the way to the post-event automatic settlement and reporting.

It simply provides everything an event organizer needs to successfully and efficiently manage their event without the need to worry about any technicalities or staff management.
NFCtron not only provides the system as a service, but also experienced Event Managers, cashiers, check-in brigadiers, and other staff to operate the system and the event itself.
This allows organizers to focus on the event communication, marketing, line-ups, and other important aspects of the event.

Put, NFCtron offers organizers a peace of mind and a guarantee that their event will be a success.

\subsection*{NFCtron Company}
\label{subsec:introduction-background-motivation-nfctron}
NFCtron is a Czech company that has been operating since around 2019.
In its early beginning during a COVID-19 pandemic was on the verge of survival because of the event industry being paralyzed by the government restrictions.
However, the company survived and even in these challenging times managed to turn the disadvantage into an advantage by focusing on the core system and product development\cite{nfctron_en_about_us}.

Years later, the system became robust and reliable.
It is now used by many event organizers across the Czech Republic and Slovakia and is currently expanding to other countries in Central Europe such as Austria\cite{nfctron_blog_nfctron_austria_events_festivals_september_visitors_payments}, Poland, and Germany.
In its primary market – the Czech Republic – NFCtron penetrated the market and is now the leading cashless payment system provider for festivals and other events.

In recent years, the company has also been focusing on expanding to the Payments market, focusing both on Card Acquiring and Card Issuing.
From the acquiring part, it is now actively developing its own SoftPOS solution that will allow vendors to accept payments via mobile phones.
On the other hand, on the Issuing part, it is also working on Card Issuing; that will allow the company to issue its own NFCtron branded payment cards in cooperation with Mastercard\cite{nfctron_blog_nfctron_keynote_mastercard_cashless_event}.

A big part of the successful market penetration was the company's focus on the business-to-business (B2B) side with the event organizers.
Giving the organizers amounts of data improving their decision-making and providing them with insights allowing them to optimize their events.
And most importantly, providing them with economic aspects and cashflow optimizations that allow many events and festivals to survive and continue to operate.

\subsection*{Personal Position and Motivation}
\label{subsec:introduction-background-motivation}
I have been with the company from the COVID-19 times.
My current position is a Chief Product Officer (CPO), and I am responsible for the product development and the product management of all the products and services that NFCtron offers.
This allows me to have a deep insight into the system and most importantly, access to all the data that the system generates.
As previously mentioned, the company's success is based on the B2B side of the business, and that means services and products provided to the event organizers.

The main product that organizers have access to is the platform called \textbf{NFCtron Hub}.
My personal goal and personal motivation to work on this thesis is to discover new ways to improve the platform and provide even more valuable insights to the organizers.

\section*{Problem Statement}
\addcontentsline{toc}{section}{Problem Statement}
\label{sec:introduction-problem-statement}
Even though the system provides a lot of data, it still has a lot of potential to provide even more valuable insights to the organizers.
Currently, the previously mentioned B2B platform, \textbf{NFCtron Hub}, provides a real-time data analytics dashboard presenting the most important KPIs and metrics to the organizers.

These KPIs and metrics summarize:
\begin{itemize}
	\item \textbf{Total sales}: the total amount spent on online ticket sales and on-site payments.
	\item \textbf{Total sales in time}: the total amount split into time intervals.
	\item \textbf{Total refunds}: the total number of sale reversals or refunds made including refunds from online tickets, refunds of on-site payments, and chip credit refunds.
	\item \textbf{Chip balances}: the current balance topped-up on the NFC chip bracelets on-site or pre-topped-up online.
	\item \textbf{Customer orders rating}: customers can rate their orders via an NFCtron mobile application, which provides the organizers with feedback on the vendor's performance and the quality of the products sold.
\end{itemize}

Moreover, it provides less clear data such as
\begin{itemize}
	\item \textbf{List of vendors}: the list of vendors presents at the event with their sales and rating.
	\item \textbf{List of products}: the list of products sold at the event with their sales and rating.
	\item \textbf{List of sale places}: the list of sale places at the event with their sales and rating.
	\item \textbf{List of top-up places}: the list of top-up places at the event with the number of top-ups made.
	\item \textbf{List of customer chips}: the list of unique individual NFC chips issued to the customers with their balance, spending, and security status.
	\item \textbf{List of customer ratings}: the list of customer ratings with their feedback from points of sale.
\end{itemize}

And finally, it also provides unstructured data in the form of data exports in tabular format that can be used for further analysis:
\begin{itemize}
	\item \textbf{Product exports} – list of all products sold with summarized metrics.
	\item \textbf{Sale places exports} – list of all sale places with summarized metrics.
	\item \textbf{Vendor exports} – list of all vendors with summarized metrics.
	\item \textbf{Deal exports} – list of summarized sales made under a deal\footnote{
		A deal is an arrangment between the organizer and the vendor that states the terms of the vendor's presence at the event, the products they are allowed to sell, the price of the products, the commission the vendor pays to the organizer and other terms of the deal.
	}~between the organizer and the vendor.
	\item \textbf{Transaction exports} – a heavy export of all transactions made at the event.
	\item \textbf{Ticket redeems exports} – list of all tickets redeemed at the event.
	\item \textbf{And other exports regarding the online sales} – list of all online ticket sales, top-ups, receipts, customers, and other data.
\end{itemize}

With the above giving some initial picture about the capabilities of the system, the platform (and thus the organizers) still face several problems or challenges that need to be addressed:

\textbf{Problem 1}: The main KPI metrics may provide some core insights, but the other less or unstructured data is not used to its potential.

\textbf{Problem 2}: The organizers are not data scientists and need a simple and clear way to understand the data.

\textbf{Problem 3}: Even with all this data, there is still a lot that can be done to dig deeper and provide more valuable insights.

\section*{Objectives of the Work}
\addcontentsline{toc}{section}{Objectives of the Work}
\label{sec:introduction-objectives}
With the problems stated above, the main goal of this thesis is to analyze, answer, and present results to important questions about the available data and the potential insights that can be extracted from it.

But to achieve this, it was a prerequisite to find a willing event organizer that would provide the data and would be willing to cooperate on the project.
Cooperate in terms of providing valuable insights into what they would like to know more about their event.

\subsection*{In Cooperation with the Event Organizer}
\label{subsec:introduction-objectives-cooperation}
For this purpose, I chose an undisclosed event organizer who has been a close and helpful partner of NFCtron for many seasons.
Together with the organizer in the first step, we have stated the following requirements to perform the analysis:
\begin{itemize}
	\item \textbf{Requirement 1}: The event and organizer should be kept undisclosed.
	\item \textbf{Requirement 2}: The data should be anonymized to not leak any possible sensitive information about vendors or customers.
\end{itemize}

The next step was to choose an event from which the data will be used.
As it cannot be disclosed any further, we will refer to the event as~\theEvent~and the organizer as~\theOrganizer.

Now the important information about~\theEvent~for this study is the following:
\begin{itemize}
	\item \theEvent~is a music festival that has been organized for several years now.
	\item \theEvent~took place in the Czech Republic at the beginning of July 2024.
	\item \theEvent~is a 3-day event with multiple stages and multiple vendors.
	\item \theEvent~uses NFCtron system for cashless payments and access control.
	\item \theEvent~had around 7,000 attendees in 2023 and had a roughly 43\% increase in 2024 to around 10,000 attendees.
\end{itemize}

\subsection*{Data Analysis Objectives}
\label{subsec:introduction-objectives-data-analysis}

The final step was to define questions or data analysis objectives that should be answered or achieved by the end of the thesis.

Together in the cooperation with~\theOrganizer~and several internal colleagues in NFCtron, we have defined the following questions for the data analysis:

\subsubsection*{Cashflow and Revenue}
\defresearchq{cashflow-total-revenue}{What was the organizer's total revenue and what did it consist of?}
\defresearchq{cashflow-top-up-balance}{How much and how was the balance topped up on the chips?}
\defresearchq{cashflow-remaining-balance}{What is the remaining balance on all chips after the event and refunds?}
\defresearchq{cashflow-total-sales}{What was the total sales of the event, how much of it was sold by the organizer, and how many external vendors?}
\begin{flushleft}
	\begin{itemize}
		\item \textit{\rqfull{cashflow-total-revenue}}
		\item \textit{\rqfull{cashflow-top-up-balance}}
		\item \textit{\rqfull{cashflow-remaining-balance}}
		\item \textit{\rqfull{cashflow-total-sales}}
	\end{itemize}
\end{flushleft}

These questions should shed light on the event's cash flow and revenue sources, which the platform does not currently cover in detail.
Possible answers to these questions may provide the organizer with valuable insights into the event's economic aspects, allowing them to optimize cashflow and revenue sources for the following event.

\subsubsection*{Performance}
\defresearchq{performance-transactions}{How many transactions were processed in total, and when was the system's largest \enquote{peak} in transaction volume?}
\defresearchq{performance-processing-during-peaks}{What was the average transaction processing time during peak hours?}
\defresearchq{performance-delays-downtimes}{Were there any significant delays or downtimes in processing transactions?}
\defresearchq{performance-best-sale-points}{What were the best sale places?}
\defresearchq{performance-best-top-up-points}{What were the best top-up points?}
\defresearchq{performance-best-vendors}{Who were the best vendors?}
\defresearchq{performance-best-products}{What were the best products?}
\begin{flushleft}
	\begin{itemize}
		\item \textit{\rqfull{performance-transactions}}
		\item \textit{\rqfull{performance-processing-during-peaks}}
		\item \textit{\rqfull{performance-delays-downtimes}}
		\item \textit{\rqfull{performance-best-sale-points}}
		\item \textit{\rqfull{performance-best-top-up-points}}
		\item \textit{\rqfull{performance-best-vendors}}
		\item \textit{\rqfull{performance-best-products}}
	\end{itemize}
\end{flushleft}

The current platform already provides some performance metrics, but these questions should provide more detailed insights into the performance of the event.

\subsubsection*{Beverage Consumption}
\defresearchq{beverage-total-consumption}{What was the total amount of beverage consumed?}
\defresearchq{beverage-returnable-cups}{How many returnable cups were issued and returned/not returned?}
\defresearchq{beverage-popular-category}{What was the most popular beverage category?}
\defresearchq{beverage-top-beer}{What was the top beer brand, and how much was consumed and sold?}
\defresearchq{beverage-top-alcoholic}{What was the most popular brand of other alcoholic beverages, and how much was consumed and sold?}
\defresearchq{beverage-top-non-alcoholic}{What was the most popular non-alcoholic beverage brand, and how much was sold?}
\begin{flushleft}
	\begin{itemize}
		\item \textit{\rqfull{beverage-total-consumption}}
		\item \textit{\rqfull{beverage-returnable-cups}}
		\item \textit{\rqfull{beverage-popular-category}}
		\item \textit{\rqfull{beverage-top-beer}}
		\item \textit{\rqfull{beverage-top-alcoholic}}
		\item \textit{\rqfull{beverage-top-non-alcoholic}}
	\end{itemize}
\end{flushleft}

Currently, a more in-depth product analysis is missing in the platform, and the most important part of the product sales analysis at festivals is the beverage consumption.
These questions should try to answer and give detailed insights into the beverage consumption, preferences, and sales at the event.

\subsubsection*{Customers}
\defresearchq{customers-total-attendance}{What was the total attendance at the event, and how many active customers were there each day?}
\defresearchq{customers-top-up-online}{How many customers topped up their credit in advance online?}
\defresearchq{customers-distribution-types}{What was the distribution of customers by type (on-site, online, staff, guest, VIP)?}
\defresearchq{customers-mobile-app}{How many customers used the mobile app?}
\defresearchq{customers-bank-cards}{What is the distribution of target banks used to refund credit?}
\defresearchq{customers-card-schemes}{What is the distribution of card schemes used to top up credit both on-site and online?}
\defresearchq{customers-new-visitors}{What was the course of the event in terms of new visitors? And when were the largest "peaks"?}
\defresearchq{customers-visitor-time}{What is the average time of a visitor from arrival to first transaction?}
\defresearchq{customers-top-up-peaks}{What was the course of the event in terms of topping up credit on-site? And when were the largest "peaks"?}
\defresearchq{customers-top-up-frequency}{What was the customer's onsite credit top-up frequency?}
\defresearchq{customers-drink-preferences}{What were the beverage preferences throughout the day?}
\defresearchq{customers-common-combinations}{What were the most common product combinations?}
\begin{flushleft}
	\begin{itemize}
		\item \textit{\rqfull{customers-total-attendance}}
		\item \textit{\rqfull{customers-top-up-online}}
		\item \textit{\rqfull{customers-distribution-types}}
		\item \textit{\rqfull{customers-mobile-app}}
		\item \textit{\rqfull{customers-bank-cards}}
		\item \textit{\rqfull{customers-card-schemes}}
		\item \textit{\rqfull{customers-new-visitors}}
		\item \textit{\rqfull{customers-visitor-time}}
		\item \textit{\rqfull{customers-top-up-peaks}}
		\item \textit{\rqfull{customers-top-up-frequency}}
		\item \textit{\rqfull{customers-drink-preferences}}
		\item \textit{\rqfull{customers-common-combinations}}
	\end{itemize}
\end{flushleft}

The customer analysis is the most important part of the data analysis.
It is crucial for the festival organizers to know their customer base and their behavior to optimize the event and make it more attractive for the customers.

Currently, no customer analysis, other than the customer ratings and list of customer chips, is available on the platform.

Answering these questions will possibly lead to the most valuable insights about the event's customer base and their behavior that no other platform or system currently provides.

Making these questions crucial and most valuable for the organizer and for the platform itself.

\subsection*{Technical Objectives}
\label{subsec:introduction-objectives-technical}
Answering the above questions will require a technical solution that will be able to process the data and provide the answers.

The scope of this study is not to implement a new system or a new platform and not even to implement any new changes to the existing NFCtron Hub platform.
It is to find answers to the above questions, present them in a clear and understandable way, and demonstrate the findings in the form of a simple internal dashboard prototype.

The technical goals of this study are:
\begin{itemize}
	\item Prepare, process, and analyze the data from~\theEvent.
	\item Find answers to the above questions and present the results.
	\item Implement a simple internal dashboard prototype that will demonstrate the findings.
\end{itemize}

\pagebreak[4]
\section*{Scope of the Study}
\addcontentsline{toc}{section}{Scope of the Study}
\label{sec:introduction-scope}
To ensure the feasibility and focus of the study, certain boundaries have been defined in terms of what is included and excluded from the scope of the study.

\subsection*{Included in the Scope}
\label{subsec:introduction-scope-included}
\begin{itemize}
	\item The study will focus on transactional, customer, and operational data from a specific event, referred to as~\theEvent.
	\item Key areas of analysis include cashflow, revenue sources, performance indicators, beverage consumption, and customer segmentation and behavior.
	\item The data analyzed includes pre-event (\eg~online top-ups), during-event (\eg~chip transactions, sales), and post-event data (\eg~credit refunds).
	\item A prototype dashboard will be developed using Python’s Dash and Plotly libraries to present the key insights.
	\item The dashboard is intended for internal use and post-event analysis by the event organizer.
\end{itemize}

\subsection*{Excluded from the Scope}
\label{subsec:introduction-scope-excluded}
\begin{itemize}
	\item \textbf{Real-Time Monitoring}: While the dashboard may be designed with real-time data potential, this study will focus solely on post-event analysis.
	\item \textbf{Multiple Event Comparisons}: This study is limited to the analysis of a single event (\theEvent) and does not involve comparative studies.
	\item \textbf{Data Collection}: The study does not involve the collection of new data and relies on data provided by~\theOrganizer~and the NFCtron system.
	\item \textbf{Implementation in NFCtron Hub}: The thesis focuses on analyzing data and developing a standalone prototype dashboard, not on direct integration into the NFCtron Hub platform.
\end{itemize}

\subsection*{Limitations}
\label{subsec:introduction-limitations}
\begin{itemize}
	\item \textbf{Anonymized Data}: To protect privacy, all customer and vendor data has been anonymized, which may restrict the analysis in some ways.
	\item \textbf{Single Event Focus}: Insights and recommendations are based only on data from a single event (\theEvent), which may limit broader applicability.
	\item \textbf{Time Constraints}: Due to the thesis's timeline, advanced features (e.g., predictive analytics) and technical implementations were deprioritized but will be considered for future work.
\end{itemize}

	% chapter - methodology
	%%% Data and Methodology
%%%%%%% Wording: ✅
%%%%%%% Styling: ✅
%%%%%%% References: ✅
%%%%% Grammar: ✅
%%% --------------------------------------------------------------
\chapter{Data and Methodology}
\label{ch:data-methodology}
This chapter addresses the process and challenges of local environment setup, obtaining, preparing and anonymizing the data.
Most importantly, this chapter describes and explains the data used for this research.
It also briefly describes the tools, technologies, and methods employed to answer the research questions.


\section{Environment and local setup}
\label{sec:data-methodology-environment}

To start off, we needed to set up some kind of environment where we would later work with the data.
The data we would be working with was stored in a PostgreSQL database.

Having direct access to the production database to perform the analysis was not a secure and ethical way to go.
Exporting only the necessary and raw data from the production database was the earliest thought, but we initially did not know what data we would need, and by exporting we would lose all the relations between the tables.

Therefore, we decided to set up a local database with the same structure as the production database where we can query and analyze the data safely.
The next step was to import the data from the production database to the local database.
Importing or simple cloning the full database was also not an option because only a small fraction of its subset was required.

So a deep internal analysis of the tables that were relevant to our study was performed.
This resulted in a list of total 21 tables that held the necessary data for the study and were necessary to be imported.

\subsection{Data Obtaining and Preparation}
\label{subsec:data-methodology-obtaining-preparation}

Almost every table was easily queried for the event and exported from the production database to a local CSV file.
But some tables (for example, and not surprisingly, the \textit{transaction} table with over 140k rows) were too large to be exported in one piece, so we had to split them into smaller parts.
Later these parts were joined together to a single CSV file using a simple Python script.

Since no direct access to the production database was used for the export but rather a database management tool, the export was not as fast as it could be and took a significant amount of time.
Moreover, the exported data, most importantly the timestamps, were in a different format than we needed.
And also, all numeric values were exported as a formatted string with a comma as a decimal separator.
So a data-preprocessing Python script was written to convert such invalid columns to the correct format.

\subsection{Local Database Setup}
\label{subsec:data-methodology-local-database-setup}
Then a step to set up the local PostgreSQL database was needed.
Due to the nature of this study, we wanted to keep the setup as simple as possible, so we used the default PostgreSQL installation without using any special environment using Docker or similar.
However, during this process we made a mistake and forgot that a PostgreSQL with a PostGIS extension\footnote{PostGIS is a spatial database extender for PostgreSQL object-relational database\cite{postgis_postgis_net}} was needed.
This, unfortunately, required re-setting up the database with the PostGIS extension.

The next step was to import the data from the CSV files to the local database.
For further database handling, analysis, and visualization, we used DataSpell, a Python IDE with a built-in database explorer and data visualization tools.
DataSpell was then used for the local database import, which prior to it required some necessary database relations and constraints modifications, since the data was exported without them and was not relevant for the study.

This whole process resulted in approximately 387k rows of data in the local database that were ready to be queried and analyzed.

\subsection{Local Database Modifications}
\label{subsec:data-methodology-local-database-modifications}
Before any analysis was performed, some modifications to the local database were needed due to some known limitations and missing data.

\subsubsection{Beverage Volumes}
\label{subsubsec:data-methodology-local-database-modifications-volume}
The first necessary limitation that the Beverage Consumption Analysis section heavily relied on was the missing information about beverage products volume in milliliters.
This information was crucial for the analysis, so a new column was added to the relevant product information tables.
However, the next step was to back-fill this information, which was not easily automated.

The First approach was to write a Python script that would try to find the volume information from the product name.
This worked for some products but not for all since the naming convention was not consistent.

After several attempts to automate this process, it was decided to manually fill in the missing information since only 425 products were present in the database.
Only 159 of them were of the beverage type and thus eligible for the volume information.

\subsubsection{Returnable Products}
\label{subsubsec:data-methodology-local-database-modifications-returnable}
Since one of the research questions was to analyze the returnable cups and this information was not easily available in the database, a new column was added to the product information tables.

This was a simple binary column that indicated whether the product was returnable or not.
Back-filling this information was also pretty straightforward since only one product was a returnable cup.

\subsubsection{Venue Map Visualization}
\label{subsubsec:data-methodology-local-database-modifications-venue-map}
One of the initial ideas was to visualize the venue map with the locations of the selling places, top-up service points, stages, and other important places.

This would be invaluable, the database was partially ready for this, but the data would be significantly time-consuming to back-fill and the later analysis and visualization would require more time.

Since these facts and that the process of preparing the data took place before completing the list of data analysis questions, this idea has been later abandoned.

\subsubsection{Event Program}
\label{subsubsec:data-methodology-local-database-modifications-program}
To present some time-related data and its correlation with the event program, it would require having the event program in the database.

Again, the database was ready for this, but no event program was set up since it was unnecessary for the event.
Therefore, this required getting the event program from the festival website and manually inserting it into the database.

This was manually a very time-consuming process, but it was necessary for the analysis.
For some simplification of the process, an AI tool was used to extract the data from the program schedule screenshots and instructed to prepare an SQL script that would insert the data into the database.

This seemed like a good idea, but the AI tool was initially hallucinating and made up some incorrect data.
But after several iterations, it successfully extracted the data and prepared the SQL script which was used, and the event program was successfully inserted into the database.

In the end, one can doubt that this process was faster than manual data entry, but it was a good exercise and a good example of how AI can be used to automate some processes.

\pagebreak[4]


\section{Data Anonymization}
\label{sec:data-methodology-anonymization}
The Data Anonymization process was necessary due to requirements initially set by the data provider and later by the ethical considerations.
This step was performed for the already imported data in the local database.
It required identifying the sensitive data and replacing them with anonymized values.

\begin{infobox}{What is Data Anonymization?}
	\textbf{Data anonymization} involves removing or encrypting sensitive data, including personally identifiable information (PII), protected health information (PHI), and other non-personal commercial sensitive data such as revenue or IP, from a data set.
	Its intent is to protect data subjects' privacy and confidentiality while still allowing data to be retained and used\cite{hd_data_anonymization_techniques}.
\end{infobox}

In this case, the most sensitive data were:
\begin{itemize}
	\item \textbf{Vendor names}: Since it included the legal names of the vendors, it was necessary to anonymize them.
	\item \textbf{Selling places}: Some selling places were named after the vendors, so it was necessary to anonymize them as well.
	\item \textbf{Customer information}: Some tables included customer information like names, emails, phone numbers, etc.
\end{itemize}

The process could have been done various ways, but the fact that this study will not be exposing internal database structure, it was decided to perform the anonymization directly in the database.

However, if one-way anonymization were to be performed, it would permanently overwrite the original data, losing the possibility to switch from anonymized to original data.
Therefore, a two-way anonymization process was chosen and performed.

This was particularly useful during the analysis phase, where the results would contain the original data for better understanding, fact-checking and for the internal presentation and consultations with the organizer.

It was done on the database level, where two new internal tables were introduced~–~\textit{public.anonymization\_config} and \textit{public.original\_values}.

Where the \textit{public.anonymization\_config} table held the configuration about which schema, table, and column should be anonymized and how.
For the usage, a simple SQL function was created to define the anonymization configuration in a simple JSON format that looked like in the~\autoref{lst:anonymization-configuration}.

\begin{listing}[h]
	\begin{minted}{sql}
		SELECT configure_anonymization('[
			{ "table": "schema.seller", "columns": ["legal_name", "name"] },
			...
			{ "table": "schema.user_account", "columns": ["email", "last_name", "first_name", "phone"] },
			]'::JSONB
		);
	\end{minted}
	\caption{Anonymization configuration example}
	\label{lst:anonymization-configuration}
\end{listing}

The \textit{public.original\_values} table was used to store the original values of the anonymized columns.
Again, using a simple SQL function \textit{anonymize\_database()}, it would store the original values and anonymize the configured table columns.

One particular challenge was anonymizing the values smartly.
It could have been easily done by replacing the values with random strings, hashes, or encrypted values.
But working with data where a vendor is named \textit{fa65165b923e9cc} is not very convenient.

Therefore, a simple SQL function was written to anonymize the value depending on the configuration.
This allowed configuring the anonymization to:
\begin{itemize}
	\item replace vendor names with values like \textit{Vendor 1},
	\item customer emails with \textit{03b09592-d0eb-43a3-9941-30d38ade6bce@gmail.com} keeping the original email domain,
	\item selling places with values like \textit{Place 1} where the original name contained sensitive information, but keep original values for places like \textit{BAR L2},~etc.
\end{itemize}

In the end, it resulted in a database with anonymized values per stated configuration that could have been used for the analysis and results presentation safely.
However, the original values were still present in the database, and the database could have been anytime easily restored to the original state if needed and vice versa.


\section{Data Structure}
\label{sec:data-methodology-structure}
Without exposing the internal database structure, the abstract data structure this study has been working with, is described in this section.

\subsection{Financial Data}
\label{subsec:data-methodology-structure-financial}
\textbf{Transactions} (approx.\ 300k rows): Transactional data that holds the information about the type, amount, timestamps, and links to products, places, and other related entities.
This analysis relies on and works with several transaction types, including \textbf{top-up charge} and \textbf{refund} transactions~\footnote{Top-up transactions mean funding or refunding chip credit balances.},
\textbf{order sales} and \textbf{refund} transactions~\footnote{Order transactions mean spending the credit balance for products.}~and
\textbf{chip registrations}\footnote{Chip registrations mean records of when the system registered the chip bracelets}.

\textbf{Credit Refunds} (approx.\ 15k rows): Post-event credit refund requested by the customers with the information about the amount, timestamp, customer, and the bank account to which the refund was sent.
This data enables our analysis to correctly handle disposable credit balances, more information about the anonymous customers~\footnote{Anonymous customer essentially means a user without a registered account}~and their behavior.

\begin{blue-box}{Why is it important?}
	These records also form the backbone of the analysis, enabling insights into revenue, sales trends, and customer behavior.
\end{blue-box}

\subsection{Customer Data}
\label{subsec:data-methodology-structure-customer}

\textbf{User Accounts} (approx.\ 5k rows): Registered customer accounts with the information about the user and its potential online order history and other related information.

\textbf{Tickets and Orders} (approx.\ 30k rows): Information about the online sold tickets, its types, prices, timestamps, online order-related information.

\begin{blue-box}{Why is it important?}
	With the above data, this analysis can work with more customer information supporting the customer behavior analysis, customer segmentation, and other related analysis.
\end{blue-box}

\subsection{Event Data}
\label{subsec:data-methodology-structure-event}

\textbf{Places} (approx.\ 400 rows): Selling and top-up service points, zones for access control and its other relations.

\textbf{Products} (approx.\ 500 rows): Essentially a product catalog including the product name, price, category, volume, seller ownership, and seller-organizer deal-related links.

Important information about products is their supported categorization that will later be used for the sales analysis and can be seen in~\autoref{tab:product-categories}.

\begin{table}[htbp]
	\centering
	\footnotesize
	% @formatter:off
	\begin{tabularx}{\textwidth}{|>{\columncolor{unicorn_blue!5}}X|>{\columncolor{unicorn_blue!5}}l|}
		\hline
		\rowcolor{unicorn_blue}
		\textbf{\color{white} Category} & \textbf{\color{white} Description} \\
		\hline
		\hline
		Nonalcoholic & Any non-alcoholic beverages (e.g., coffee, water, etc.) \\
		Beer & Any kind of beer \\
		Wine & Any kind of wine \\
		Other Alcohol & Any other kind of alcoholic beverages (e.g., shots, cocktails, etc.) \\
		Salty & Any salty snacks \\
		Sweet & Any sweet snacks \\
		Other & Any other products that do not fit into the above categories \\
		\hline
	\end{tabularx}
	% @formatter:on
	\caption{Product categories}
	\label{tab:product-categories}
	\source
\end{table}

\textbf{Event Program} (approx.\ 140 rows): Event program schedule with the information about the stages, performers, times and other related information.

\begin{blue-box}{Why is it important?}
	This data provides more context to the event when combined with the financial and customer data above.
	Enabling the analysis to work with the event program, its correlation with the sales, customer behavior, and other related analysis.
\end{blue-box}

\subsection{Data Processing Views}
\label{subsec:data-methodology-structure-views}
To efficiently query the studied data during the analysis, several SQL views and functions were created to simplify and speed up the process.

\textbf{Transaction Commission Calculation}: A function that calculates the commission for each transaction based on the product and the seller-organizer deal.
This was a crucial method required to calculate and analyze the commission from event order sales contributing to the organizer's revenue.

\textbf{Transaction Enrichment}: Since the transactional data consisted of several transaction types which were not easily distinguishable, a view was created to enrich the transaction data with the transaction type information.
It also benefited from the transaction commission calculation function mentioned above, which helped to easily calculate the commission for each transaction.

\textbf{Chip Customers}: Probably the most complex function that returns the customers at the event.
Since the transactional data is architected using an Event Sourcing pattern
\footnote{Event Sourcing is a pattern where the state of the system is determined by a sequence of events\cite{aw_implementing_event_sourcing_using_a_relational_database}.}, the customer information
is not directly available and needs to be compiled from the transactional history.
This function was constructed in a way where it supports time-based filtering and provides extensive insights into the customers, which is shown in~\autoref{tab:chip-customers-columns}.

\begin{table}[htbp]
	\centering
	\footnotesize
	% @formatter:off
	\begin{tabularx}{\textwidth}{|>{\columncolor{unicorn_blue!5}}X|>{\columncolor{unicorn_blue!5}}l|}
		\hline
		\rowcolor{unicorn_blue}
		\textbf{\color{white} Column Name} & \textbf{\color{white} Description} \\
		\hline
		\hline
		CHIP\_ID & Unique chip identifier \\
		CHIP\_TYPE & Type of chip (\eg~regular, VIP, online, staff) \\
		REG\_AT & Timestamp of chip registration \\
		FIRST\_TRX & First transaction timestamp associated with the chip \\
		LAST\_TRX & Last transaction timestamp associated with the chip \\
		LAST\_BALANCE & Last known balance at the specified time frame \\
		ACTUAL\_BALANCE & Balance at the specified time frame after credit refunds \\
		\hline
		IS\_BLOCKED & Indicates if the chip is blocked due to suspicious activity \\
		HOURS\_ACTIVE & Total active hours of the chip (daily sum) \\
		DAYS\_ACTIVE & Total number of days the chip was active \\
		\hline
		T\_COUNT & Total number of transactions associated with the chip \\
		O\_TOTAL\_CNT & Total number of orders placed using the chip \\
		O\_TOTAL\_AMT & Total amount spent through orders \\
		O\_MAX\_AMT & Maximum amount spent in a single order \\
		O\_AVG\_AMT & Average amount spent per order \\
		O\_MODE\_AMT & Most common amount spent per order \\
		OS\_AVG\_AMT & Average amount spent on sales orders (excluding refunds) \\
		OS\_MODE\_AMT & Most common sale amount (excluding refunds) \\
		TU\_TOTAL\_CNT & Total number of top-ups made to the chip \\
		TU\_TOTAL\_AMT & Total amount credited to the chip via top-ups \\
		TU\_MAX\_AMT & Maximum amount credited in a single top-up \\
		TU\_AVG\_AMT & Average amount credited per top-up \\
		TU\_MODE\_AMT & Most common amount credited per top-up \\
		TU\_CARD\_BRAND & Used card brand for top-ups (\eg~Visa, Mastercard) \\
		\hline
		BR\_AMT & Total amount refunded to the customer's bank account \\
		BR\_EMAIL\_DOMAIN & Domain of the refund request email (\eg~gmail.com) \\
		BR\_COUNTRY & Country associated with the bank account for refund \\
		BR\_REQ\_SOURCE & Source of the refund request (\eg~iOS, Android, Web) \\
		BR\_BANK\_NAME & Name of the Czech bank used for the refund \\
		BR\_CREATED & Timestamp when the refund request was created \\
		BR\_APPROVED & Timestamp when the refund request was approved \\
		\hline
		A\_EMAIL\_DOMAIN & Email domain of the account \\
		A\_COUNTRY\_NAME & Country associated with the account \\
		A\_REQ\_SOURCE & Source of the account creation (\eg~iOS, Android, Web) \\
		EO\_PAYMENT\_METHOD & Payment method for orders (\eg~card, bank transfer) \\
		EO\_CARD\_BRAND & Used card brand for online order (Visa, Mastercard \\
		EO\_REQ\_SOURCE & Source of the order request (\eg~iOS, Android, Web) \\
		\hline
	\end{tabularx}
	% @formatter:on
	\caption{Customer chips function return table}
	\label{tab:chip-customers-columns}
	\source
\end{table}

For further understanding, an enumeration of chip types should be mentioned; that can be seen in~\autoref{tab:chip-types}.

\begin{table}[H]
	\centering
	\footnotesize
	% @formatter:off
	\begin{tabularx}{\textwidth}{|>{\columncolor{unicorn_blue!5}}X|>{\columncolor{unicorn_blue!5}}l|}
		\hline
		\rowcolor{unicorn_blue}
		\textbf{\color{white} Chip Type} & \textbf{\color{white} Description} \\
		\hline
		\hline
		Regular & Chip issued at the event without any prior credit \\
		VIP & VIP chip issued with artificial credit on behalf of the organizer \\
		Online & Chip issued via online purchase or top-up \\
		Staff & Chip issued to the staff members \\
		Guest & Chip issued to the guests (band members, etc.) \\
		\hline
	\end{tabularx}
	% @formatter:on
	\caption{Chip types enumeration}
	\label{tab:chip-types}
	\source
\end{table}


\section{Tools and Technologies}
\label{sec:data-methodology-tools}

As mentioned earlier, this process drew from a variety of tools and technologies to handle the data and prepare for the analysis.
These main tools and technologies included:
\begin{itemize}
	\item \textbf{PostgreSQL}: An open-source relational database management system used to store and query the data\footnote{\url{https://www.postgresql.org/}}.
	\item \textbf{PostGIS}: An extension of PostgreSQL that supports geospatial data, enabling spatial analysis and visualization\footnote{\url{https://postgis.net/}}.
	\item \textbf{DataSpell}: An integrated development environment (IDE) for data science and analytics, used for database exploration and data visualization\footnote{\url{https://www.jetbrains.com/dataspell/}}.
	\item \textbf{Python}: Used for data preprocessing, querying, and analysis, along with libraries like Pandas and Matplotlib for data manipulation\footnote{\url{https://www.python.org/}, \url{https://pandas.pydata.org/}, \url{https://matplotlib.org/}}.
	\item \textbf{Claude AI}: Utilized for extracting data from unstructured sources (e.g., program schedules) and automating repetitive tasks like data entry\footnote{\url{https://claude.ai/}}.
\end{itemize}

Using such tools during this process provided a convenient environment for data handling and processing, initial analysis and ensuring the accuracy and efficiency for further analysis and results presentation.


\section{Conclusion}
\label{sec:data-methodology-conclusion}
This chapter laid out the process of setting up the local environment, collecting and preparing the data, anonymizing sensitive information, and introducing the data structure.

Some key challenges, such as missing beverage volume information, data anonymization, and event program integration, were addressed through a combination of manual and automated processes.

In the end, the use of SQL views for data enrichment and functions for further data processing ensured efficient querying and analysis.
Hopefully, it also laid the groundwork for answering the research questions and achieving the study's goals, as detailed in the later chapters.

For better understanding and visualization of this initial comprehensive Knowledge Data Discover process\footnote{Knowledge Data Discovery (KDD) is the process of discovering useful knowledge from a collection of data\cite{uord_kdd_1_kdd}}, a simplified diagram, which can be seen in the~\autoref{fig:data-methodology-diagram}, was created to illustrate the workflow of the process.

\begin{figure}[h]
	\centering
	\includesvg[width=0.875\textwidth]{\ThesisFigures/diagrams/kdd-workflow-paths.svg}
	\caption{Knowledge Data Discovery workflow diagram}
	\label{fig:data-methodology-diagram}
\end{figure}

	% chapter - data analysis and results
	%%% Data Analysis and Results
%%%%%%% Wording: ⏳
%%%%%%% Styling: ⏳
%%%%%%% References: ⏳
%%%%% Grammar: ⏳
%%% --------------------------------------------------------------
\chapter{Data Analysis and Results}
\label{ch:data-analysis-and-results}

This chapter presents the data analysis and results of the research.
The goal is to address the research questions outlined earlier, with a focus on providing actionable insights for the event organizer.

The chapter is divided into several sections corresponding to key analytical areas:
\begin{enumerate}
	\item \fullref{sec:analysis-cashflow-and-revenue-sources},
	\item \fullref{sec:analysis-performance-indicators},
	\item \fullref{sec:analysis-beverage-consumption},
	\item and~\fullref{sec:analysis-customers}
\end{enumerate}

Each section focuses on a different aspect of the data analysis trying to answer the research questions, present quantitative results, visualizations, and interpretations.


\section{Cashflow and Revenue Sources Analysis}
\label{sec:analysis-cashflow-and-revenue-sources}

This section provides a comprehensive view of the festival's financial performance and cash flows.
It should answer critical questions about how finances were funded into the system, how were they processed, and what were the final outcomes.

For this analysis, four questions were previously formulated.
However, they were reordered to better fit the narrative of the analysis and logical flow of the chapter:
\begin{enumerate}
	\item \textit{\researchq{cashflow-top-up-balance}}
	\item \textit{\researchq{cashflow-total-sales}}
	\item \textit{\researchq{cashflow-remaining-balance}}
	\item \textit{\researchq{cashflow-total-revenue}}
\end{enumerate}

In the end, this section should provide a clear picture of the financial flows during the event and easy understanding of the generated revenue from various sources.

\subsection{Chip Top-Up Analysis}
\label{subsec:analysis-chip-top-up}
\begin{gray-box}{Research Question}
	\textit{\researchq{cashflow-top-up-balance}}
\end{gray-box}

Attendees could top up their chip balances via online prepayments or on-site using cash or card.
Additionally, the system allows to top-up~\enquote{artificial} credit for VIP-issued chips which is also a mean of funding the system.
However, these VIP credits are later not refundable, but this will be discussed in the next section.

This subsection quantifies these methods, highlighting their respective contributions to the overall top-up total.

To get the results, it was necessary to find all top-up transactions and their respective payment methods used.
This resulted in \bfmtnum{17704}~top-up transactions, with a total value of~\bfmtczk{14520973}.

When looking at the grouping by payment methods, the results in~\autoref{fig:top-up-transactions-by-payment-method} give a clear picture of the distribution.

\begin{figure}[H]
	\centering
	% First the pie chart
	\includegraphics[width=0.75\textwidth]{\ThesisFigures/charts/topup-methods}
	\vspace{1em}  % add some space between chart and table

	% Then the table
	\small
	\begin{tabular}{@{}lrr@{}}
		\toprule
		\textbf{Payment Method}                & \textbf{Count} & \textbf{Total Value (CZK)} \\
		\midrule
		\colorindicator{chart2}Card terminal     & \fmtnum{8486}  & \fmtczk{7264503}           \\
		\colorindicator{chart3}Cash              & \fmtnum{7561}  & \fmtczk{5782570}           \\
		\colorindicator{chart1}Online pre top-up & \fmtnum{1634}  & \fmtczk{1436400}           \\
		\colorindicator{chart4}VIP issued        & \fmtnum{23}    & \fmtczk{37500}             \\
		\bottomrule
	\end{tabular}
	\caption{Top-Up Transactions by Payment Method}
	\label{fig:top-up-transactions-by-payment-method}
\end{figure}

Thanks to the results, it is clear how many funds did the system receive and by what means.

\begin{blue-box}{Key Takeaways}
	\begin{itemize}
		\item Total top-up amount was~\bfmtczk{14520973}.
		\item Most used payment method was card terminal at the event with 50\% of all top-ups.
		\item Only around 10\% of the top-ups were done online.
	\end{itemize}
\end{blue-box}

\subsection{Sales Analysis}
\label{subsec:analysis-sales}
\begin{gray-box}{Research Question}
	\textit{\researchq{cashflow-total-sales}}
\end{gray-box}

The sales analysis was crucial for understanding the overall sales behavior and served as a basis for further insights tightly connected to the revenue sources.

To answer the research question, it was necessary to find all sales transactions and their respective sellers and to divide them into two groups: the direct organizer's sales and external vendors' sales.
And for better understanding, the sales were also grouped by the product categories (see~\autoref{tab:product-categories} for the list of categories).

The results show that the total sales of the event were~\bfmtczk{11711807} with the organizer's sales being~\bfmtczk{8240264} and the external vendors' sales~\bfmtczk{3471543}.

The organizer, most importantly, sold all the beer beverages and most of the non-alcoholic and alcoholic (spirits) beverages.
Whereas the external vendors sold mainly the food, wine beverages and other uncategorized products.
This can be seen in~\autoref{fig:sales-organizer-vs-vendors} below.

\begin{figure}[H]
	\centering
	\includegraphics[width=0.99\textwidth]{\ThesisFigures/charts/sales-vendors}
	\caption{Sales of the Organizer vs. External Vendors}
	\label{fig:sales-organizer-vs-vendors}
\end{figure}

The organizer also sold not so little of uncategorized products, which after further investigation turned out to be ticket sales at the event amounting to~\bfmtczk{684700}.

In total, the organizer direct sales were \textbf{70\%} of the total sales, which is a significant portion, and thus the organizer itself has even bigger influence on the event's financial performance.

\begin{blue-box}{Key Takeaways}
	\begin{itemize}
		\item Total sales of the event were~\bfmtczk{11711807}, where organizer sales were~\textbf{70\%} of the total.
		\item The organizer sold all beer beverages and the majority of the non-alcoholic and alcoholic beverages.
		\item The organizer also sold tickets at the event amounting to~\bfmtczk{684700}.
		\item External vendors sold mainly food, wine beverages, and other uncategorized products.
	\end{itemize}
\end{blue-box}

\todo{Better chart} % TODO

\subsection{Remaining Chip Balances}
\label{subsec:analysis-remaining-balances}
\begin{gray-box}{Research Question}
	\textit{\researchq{cashflow-remaining-balance}}
\end{gray-box}

The remaining chip balances are crucial for the event organizer as they represent the potential revenue that can be still claimed.
Any unclaimed balances after a given refund period, which is usually up to 14~days after the event will be considered as organizer's taxable revenue.

Out of the total top-up amount of~\bfmtczk{14520973}, the total spent credit amounted to~\bfmtczk{10984945}, which left a total of~\bfmtczk{3536028} on the chips before refunds.
After refunds – done both at the event (\bfmtczk{15379}) and later via online bank refund requests (\bfmtczk{3163567}) – the remaining balance was reduced to~\bfmtczk{357082}.

However, this still included the artificially issued VIP credits with leftover balance of~\bfmtczk{12405}.
The system also reported integrity errors in the data, which resulted in a total of~\bfmtczk{10246} due to fraudulent activities performed by some attendees which were automatically suspended by the system.

This left the total unclaimed balance at~\bfmtczk{334431}, which has been claimed by the organizer as taxable revenue.

Since these numbers can be quite abstract, the results in a form of sankey diagram in~\autoref{fig:remaining-balances-sankey} below provide a clear picture of the flow of the funds.

\todo{Better chart} % TODO
\begin{figure}[H]
	\centering
	\includegraphics[width=0.99\textwidth]{\ThesisFigures/charts/balances-sankey}
	\caption{Remaining Chip Balances Sankey Diagram}
	\label{fig:remaining-balances-sankey}
\end{figure}

Thanks to this breakdown, it is clear how the remaining balances were reduced and what was the final outcome.
These results are important for the last part of this section, which is the total revenue of the organizer.

\begin{blue-box}{Key Takeaways}
	\begin{itemize}
		\item Total unused credit was~\bfmtczk{3536028}.
		\item Credit refunded to customers was~\bfmtczk{3178946}.
		\item After VIP issued credits and system integrity error, the unclaimed balance was~\bfmtczk{334431}.
	\end{itemize}
\end{blue-box}

\subsection{Total Revenue of the Organizer}
\label{subsec:analysis-total-revenue}
\begin{gray-box}{Research Question}
	\textit{\researchq{cashflow-total-revenue}}
\end{gray-box}

The festival's financial model is based on a combination of revenue streams.

The most important stream is the \textbf{commission from the vendor sales}, which is arranged in advance between the organizer and the vendors.
The commission is, in this case, a percentage (ranging from 15\% to 30\% depending on the deal) of the vendor sales amount without VAT\@.

Therefore, this required finding all sales transactions made at the external vendors' stands and calculating the commission based on the agreed percentage.
However, this was not a straightforward task, since a transaction could contain multiple products even from different vendors.

This required a more complex calculation, for which was used the previously mentioned data processing views which were designed for this purpose.
In the end, the total revenue from sales commissions was~\bfmtczkp[2]{820712,79}.

Another source of revenue is the \textbf{unclaimed chip balances}, which, after a credit refund period, are considered as taxable revenue for the organizer.
This, thanks to the previous subsection, was found to be~\bfmtczk{334431}.

\todo{Better chart} % TODO
\begin{figure}[H]
	\centering
	\includegraphics[width=0.48\textwidth]{\ThesisFigures/charts/revenue-sources-direct}
	\caption{Breakdown of Direct Revenue Streams}
	\label{fig:revenue-breakdown-direct}
\end{figure}

Currently totalling~\bfmtczkp[2]{1155143,79} is the direct revenue of the organizer from the event and can be seen in~\autoref{fig:revenue-breakdown-direct} above.

However, given the circumstances and setup of this event, there were also additional, but indirect revenue streams that were not included in the total revenue.
These include \textbf{the online ticket sales}, which were sold by the organizer and \textbf{the direct sales of the organizer}.
They were not included in the total direct revenue, as they may misinterpret the results since the analysis lacks expenses of the organizer.

If we were to include these, the total revenue would increase by~\bfmtczk{11179700} from the online ticket sales and~\bfmtczk{8240264} from the direct sales, which would result in a total revenue of~\bfmtczkp[2]{20575107,79}.

To better understand the revenue streams, the results are visualized in~\autoref{tab:revenue-summary-breakdown} and in~\autoref{fig:revenue-breakdown-total} below.

\begin{table}[H]
	\centering
	\begin{tabularx}{\textwidth}{|>{\columncolor{unicorn_blue!5}}X|>{\columncolor{unicorn_blue!5}}r|}
		\hline
		\rowcolor{unicorn_blue}
		\textbf{\color{white}Revenue Stream}         & \textbf{\color{white}Amount (CZK)} \\
		\hline
		\hline
		\colorindicator{chart1}Vendor Commissions      & \fmtczkp[2]{820712.79}             \\
		\colorindicator{chart2}Unclaimed Chip Balances & \fmtczk{334431}                    \\
		\hline
		\textbf{Total Direct Revenue}                  & \bfmtczkp[2]{1155143.79}           \\
		\hline
		\colorindicator{chart3}Online Ticket Sales     & \fmtczk{11179700}                  \\
		\colorindicator{chart4}Organizer Direct Sales  & \fmtczk{8240264}                   \\
		\hline
		\textbf{Total Revenue (All Streams)}           & \bfmtczkp[2]{20575107.79}          \\
		\hline
	\end{tabularx}
	\caption{Revenue Summary Breakdown}
	\label{tab:revenue-summary-breakdown}
\end{table}

\todo{Better chart} % TODO
\begin{figure}[H]
	\centering
	\includegraphics[width=0.75\textwidth]{\ThesisFigures/charts/revenue-sources-total}
	\caption{Breakdown of All Revenue Streams}
	\label{fig:revenue-breakdown-total}
\end{figure}

\begin{blue-box}{Key Takeaways}
	\begin{itemize}
		\item Total direct revenue of the organizer was~\bfmtczkp[2]{1155143,79}.
		\item Vendor sale commission contributed to~approximately 71\% of the total direct revenue.
		\item With other indirect revenue streams, the total revenue would be~\bfmtczkp[2]{20575107,79}.
	\end{itemize}
\end{blue-box}

\subsection{Summary}
\label{subsec:analysis-cashflow-summary}

This section provided a comprehensive view of the festival's financial performance and cash flows.
The results covered the top-up transactions, sales analysis, remaining chip balances, and the total revenue of the organizer and contributed to a better understanding from the financial perspective of the festival.

Nevertheless, results covered in these subsections are only a part of the whole picture and can be interpreted in various ways.

For this particular challenge, a summarized cash flow diagram of payments was created, containing thus only the direct revenue streams.
This diagram can be seen in the~\autoref{fig:cash-flow-diagram} below.

\begin{figure}[H]
	\centering
	\includegraphics[width=0.99\textwidth]{\ThesisFigures/charts/revenue-cash-flows}
	\caption{Overall Cash Flow Diagram}
	\label{fig:cash-flow-diagram}
\end{figure}

This diagram provides a clear overview of the financial flows during the festival and nicely summarizes the results of this analysis.

\begin{blue-box}{Key Takeaways}
	\begin{itemize}
		\item Total incoming money flow was~\bfmtczk{14520973} from top-up transactions and~\bfmtczk{726862} from non-chip sales.
		\item Total sales amounted to~\bfmtczk{11711807}.
		\item Which left a total of~\bfmtczk{3536028}~in unused credit before refunds.
		\item After refunds and non-refundable chips, the remaining balance left was~\bfmtczk{334431} claimed as taxable revenue.
		\item Commission from external vendor sales contributed to~\bfmtczkp[2]{820712,79}~of the total direct revenue.
		\item Together, the total direct revenue of the organizer was~\bfmtczkp[2]{1155143,79}.
	\end{itemize}
\end{blue-box}


\section{Performance Indicators Analysis}
\label{sec:analysis-performance-indicators}

This section focuses on the performance indicators of the event.
The goal is to identify key metrics that can be used to further evaluate the event and its success.
The potential of this analysis is to measure the~\enquote{greatness}~and the size of the event in terms of performance.

For this analysis, seven questions were previously formulated:
\begin{itemize}
	\item \textit{\researchq{performance-transactions}}
	\item \textit{\researchq{performance-processing-during-peaks}}
	\item \textit{\researchq{performance-delays-downtimes}}
	\item \textit{\researchq{performance-best-sale-points}}
	\item \textit{\researchq{performance-best-top-up-points}}
	\item \textit{\researchq{performance-best-vendors}}
	\item \textit{\researchq{performance-best-products}}
\end{itemize}

The results of this analysis should provide insights into the event's performance and help the organizer to understand the key metrics that can be used to evaluate the event's success.

To answer these questions, this section is divided into two parts:
\begin{enumerate}
	\item \fullref{subsec:analysis-performance-indicators-transactions},
	\item and~\fullref{subsec:analysis-performance-indicators-best}.
\end{enumerate}

\subsection{Transactions Processing Analysis}
\label{subsec:analysis-performance-indicators-transactions}

This subsection will focus on the processing of transactions during the event in pursuit of answering the three first research questions of this section.

\begin{gray-box}{Research Question}
	\textit{\researchq{performance-transactions}}
\end{gray-box}

This question actually consists of two sub-questions, which will be addressed separately.

The first part questions the total number of transactions processed during the event.
Which was actually pretty straightforward to answer, as the system was designed to track all transactions and their respective types.
The resulted total number of transactions was~\bfmtnum{141381} consisting of \bfmtnum{110854}~sales transactions, \bfmtnum{17726}~top-up transactions and~\bfmtnum{12801}~chip register transactions.

The second part focuses on rather time-related metrics and asks about the processing peak times during the festival.
For this part, it was necessary to spread out the above transactions over the time and find the peaks.

The results in~\autoref{fig:transactions-processing-peaks} below show the distribution of the processed transactions over time.
It clearly identifies the peak on the last day of the festival at 18:00 amounting to~\bfmtnum{8986}~transactions.

\todo{Better chart} % TODO
\begin{figure}[H]
	\centering
	\includegraphics[width=0.99\textwidth]{\ThesisFigures/charts/performance-transactions-in-time}
	\caption{Transactions Processing Peaks}
	\label{fig:transactions-processing-peaks}
\end{figure}

\begin{blue-box}{Key Takeaways}
	\begin{itemize}
		\item Total number of transactions processed was~\bfmtnum{141381} consisting mainly (7̃8\%) of order transactions.
		\item The peak of transactions was on the last day of the festival at 18:00 with~\bfmtnum{8986}~transactions processed at that hour.
	\end{itemize}
\end{blue-box}

The following two questions (\autoref{rq:performance-processing-during-peaks}~and~\autoref{rq:performance-delays-downtimes}), focus on processing times and potential delays during the event.

\begin{gray-box}{Research Question}
	\textit{\researchq{performance-processing-during-peaks}}
\end{gray-box}

The answer to this question is closely related to the previous one, as it requires the identification of the processing times during the peak times, which were already identified.

It required finding the average processing time, meaning difference between the transaction creation and its completion times.

\begin{info-box}{What causes the processing time?}
	\textit{
		Time when the transaction is created is the time when the in-place offline-supported system created the transaction, and the processed time is later when the central system receives the transaction and processes it.
		The delays can be caused by various factors, such as network latency, offline mode active, system load, or even the transaction type.
	}
\end{info-box}

The results show that the average processing time during the peak times was approximately~\bfmtnum{40}~seconds.

When slightly changing the displayed data, we also get the answer to the~\autoref{rq:performance-delays-downtimes} about the potential delays and downtimes during the event.

\begin{gray-box}{Research Question}
	\textit{\researchq{performance-delays-downtimes}}
\end{gray-box}

The chart in the~\autoref{fig:rq7-delays-in-processing}~shows the distribution of the processing times over the time and identifies one high processing peak of approximately~\bfmtnum{13}~minutes.
This is highly unusual and indicates a vendor misuse of the system or accidentally put the system into offline mode.

\todo{Better chart} % TODO
\begin{figure}[H]
	\centering
	\includegraphics[width=0.99\textwidth]{\ResultsDir/rq7-delays-in-processing}
	\caption{Transaction Processing Peaks}
	\label{fig:rq7-delays-in-processing}
\end{figure}

Other high peaks are visible on the second day in the afternoon with ~ \bfmtnum{5}~minutes processing time, which was probably caused by the initial load on that day.

\begin{blue-box}{Key Takeaways}
	\begin{itemize}
		\item The average processing time during the peak times was~\bfmtnum{40}~seconds.
		\item The highest processing peak was approximately~\bfmtnum{13}~minutes, indicating a potential misuse of the system.
		\item Other high peaks were visible on the second day in the afternoon with ~ \bfmtnum{5}~minutes processing time.
	\end{itemize}
\end{blue-box}

These results provide insights into the system's performance during the event, its reliability, and potential bottlenecks.
It also shows the festival's popularity and the system's ability to handle the load.

\subsection{Best Sale \& Top-Up Points, Vendors, and Products Analysis}
\label{subsec:analysis-performance-indicators-best}
In this subsection, the main goal will be to address the last four research questions of this section and provide insights into the best: selling points, top-up points, vendors, and products.

The problem with these question statements is that they are quite broad and can be interpreted in various ways.
What does a~\enquote{best} mean in this context?

It can be the most profitable, the best rated, the most visited, etc.
But since we are exploring the performance indicators, the best should be understood as the~\enquote{busiest}.
Which in terms of the system and this analysis should mean \textbf{the most transactions created} and the point's ability to handle the load.

\subsubsection{Best Top-Up Points}
\label{subsubsec:analysis-best-top-up-points}
The first focus will be on the best top-up points since unlike the selling points, vendors and products, the top-up points are not linked to any specific product or vendor.
\begin{gray-box}{Research Question}
	\textit{\researchq{performance-best-top-up-points}}
\end{gray-box}

To find these results, it required finding all top-up transactions, aggregate them in a bucket-like time frame and finally calculate their total counts, max peaks and averages over time.

This resulted in the following findings in the~\autoref{tab:best-topup-points} below.

\begin{table}[htbp]
	\centering
	\small
	\begin{tabularx}{\textwidth}{
		|>{\columncolor{unicorn_blue!5}\centering\arraybackslash}p{1cm}
		|>{\columncolor{unicorn_blue!5}\raggedright\arraybackslash}X
		|>{\columncolor{unicorn_blue!5}\raggedleft\arraybackslash}p{2.5cm}
		|>{\columncolor{unicorn_blue!5}\raggedleft\arraybackslash}p{2.5cm}
		|>{\columncolor{unicorn_blue!5}\raggedleft\arraybackslash}p{2.5cm}|}
		\hline
		\rowcolor{unicorn_blue}
		\textbf{}
		& \textbf{\color{white}Top-Up point}
		& \textbf{\color{white}Customers}
		& \textbf{\color{white}Transactions}
		& \textbf{\color{white}Max trx./h}
		\\\hline\hline
		% first rows
		\csvreader[
		head to column names,
		late after line={\\\hline},
		filter={\thecsvinputline<6}
		]{\ResultsDir/rq9-best-topup-points.csv}{
			entity=\colentity,
			customer_count=\colcustomers,
			transaction_count=\coltrxcount,
			max_hourly_peak=\colhourlypeak
		}{
			\the\numexpr\thecsvinputline-1
			& \colentity
			& \num[group-separator={,}]{\colcustomers}
			& \num[group-separator={,}]{\coltrxcount}
			& \num[group-separator={,}]{\colhourlypeak}
		}
		% separator
		\noalign{\vspace{1mm}}
		\multicolumn{5}{c}{\footnotesize{\textellipsis}}
		\\
		\noalign{\vspace{1mm}}
		\hline
		% middle rows
		\csvreader[
		head to column names,
		late after line={\\\hline},
		filter={\thecsvinputline>15 \AND \thecsvinputline<20}
		]{\ResultsDir/rq9-best-topup-points.csv}{
			entity=\colentity,
			customer_count=\colcustomers,
			transaction_count=\coltrxcount,
			max_hourly_peak=\colhourlypeak
		}{
			\the\numexpr\thecsvinputline-1
			& \colentity
			& \num[group-separator={,}]{\colcustomers}
			& \num[group-separator={,}]{\coltrxcount}
			& \num[group-separator={,}]{\colhourlypeak}
		}
		% separator
		\noalign{\vspace{1mm}}
		\multicolumn{5}{c}{\footnotesize{\textellipsis}}
		\\
		\noalign{\vspace{1mm}}
		\hline
		% last rows
		\csvreader[
		head to column names,
		late after line={\\\hline},
		filter={\thecsvinputline>25}
		]{\ResultsDir/rq9-best-topup-points.csv}{
			entity=\colentity,
			customer_count=\colcustomers,
			transaction_count=\coltrxcount,
			max_hourly_peak=\colhourlypeak
		}{
			\the\numexpr\thecsvinputline-1
			& \colentity
			& \num[group-separator={,}]{\colcustomers}
			& \num[group-separator={,}]{\coltrxcount}
			& \num[group-separator={,}]{\colhourlypeak}
		}
	\end{tabularx}
	\caption{Best Top-Up Points}
	\label{tab:best-topup-points}
\end{table}

This indicates that the most busy top-up points were somehow evenly distributed with approximately around \textbf{1000~transactions} processed during the event with average peaks of around \textbf{100~transactions/hour}.
The least busy top-up points were the specific ones, such as the Support tent, VIP and Accreditation points, which were not used so much for top-ups.

The overall distribution, shown in the~\autoref{fig:best-topup-points}, also shows that Top-up points were more busy than Check-in points.
That makes sense because the top-ups were done more frequently than the initial check-ins, but the check-ins were done in a more concentrated time frame.

\begin{figure}[H]
	\centering
	\includegraphics[width=0.99\textwidth]{\ResultsDir/rq9-best-topup-points}
	\caption{Best Top-Up Points}
	\label{fig:best-topup-points}
\end{figure}
\todo{Better chart} % TODO

Especially \textbf{Odbavení} point processed only~\bfmtnum{529}~transactions but peaked at~\bfmtnum{125}~transactions per hour, which was much higher than the other check-in points and even higher than the best top-up points.

\begin{blue-box}{Key Takeaways}
	\begin{itemize}
		\item The most busy top-up points processed around~\bfmtnum{1000}~transactions during the event.
		\item The average peak of the top-up points was around~\bfmtnum{100}~transactions per hour.
		\item The least busy top-up points were the specific ones, such as the Support tent, VIP and Accreditation points.
		\item Check-in points were less busy than the top-up points, but the \textbf{Odbavení} point peaked at~\bfmtnum{125}~transactions per hour.
	\end{itemize}
\end{blue-box}

\subsubsection{Best Sale Points}
\label{subsubsec:analysis-best-sale-points}

The next focus will be on the best sale points, which are the points where the most orders were created.

\begin{gray-box}{Research Question}
	\textit{\researchq{performance-best-sale-points}}
\end{gray-box}

The process of finding the best sale points was similar to the previous one, but this time it required finding all sales transactions and their respective points.

Out of total of~\bfmtnum{145}~sale points, the best place was undeniable the \textbf{L20 PIVNÍ STAN 1} with total~\bfmtnum{10114}~orders processed and the maximum peak of~\bfmtnum{840}~orders per hour.

Another interesting fact is the total unique users processed at the places.
Where again the \textbf{L20 PIVNÍ STAN 1} processed~\bfmtnum{9159}~unique customers which, out of \bfmtnum{10009}~active customers, is a significant portion (\bfmtnump[2]{91.50}{}\%) of the total.

\begin{table}[htbp]
	\centering
	\small
	\begin{tabularx}{\textwidth}{
		|>{\columncolor{unicorn_blue!5}\centering\arraybackslash}p{1cm}
		|>{\columncolor{unicorn_blue!5}\raggedright\arraybackslash}X
		|>{\columncolor{unicorn_blue!5}\raggedleft\arraybackslash}p{2.5cm}
		|>{\columncolor{unicorn_blue!5}\raggedleft\arraybackslash}p{2.5cm}
		|>{\columncolor{unicorn_blue!5}\raggedleft\arraybackslash}p{2.5cm}|}
		\hline
		\rowcolor{unicorn_blue}
		\textbf{}
		& \textbf{\color{white}Sale Point}
		& \textbf{\color{white}Customers}
		& \textbf{\color{white}Orders}
		& \textbf{\color{white}Max orders./h}
		\\\hline\hline
		% first rows
		\csvreader[
		head to column names,
		late after line={\\\hline},
		filter={\thecsvinputline<9}
		]{\ResultsDir/rq8-best-sale-points.csv}{
			entity=\colentity,
			customer_count=\colcustomers,
			transaction_count=\coltrxcount,
			max_hourly_peak=\colhourlypeak
		}{
			\the\numexpr\thecsvinputline-1
			& \colentity
			& \num[group-separator={,}]{\colcustomers}
			& \num[group-separator={,}]{\coltrxcount}
			& \num[group-separator={,}]{\colhourlypeak}
		}
		% separator
		\noalign{\vspace{1mm}}
		\multicolumn{5}{c}{\footnotesize{\textellipsis}}
		\\
		\noalign{\vspace{1mm}}
		\hline
		% last rows
		\csvreader[
		head to column names,
		late after line={\\\hline},
		filter={\thecsvinputline>132}
		]{\ResultsDir/rq8-best-sale-points.csv}{
			entity=\colentity,
			customer_count=\colcustomers,
			transaction_count=\coltrxcount,
			max_hourly_peak=\colhourlypeak
		}{
			\the\numexpr\thecsvinputline-1
			& \colentity
			& \num[group-separator={,}]{\colcustomers}
			& \num[group-separator={,}]{\coltrxcount}
			& \num[group-separator={,}]{\colhourlypeak}
		}
	\end{tabularx}
	\caption{Best Sale Points}
	\label{tab:best-sale-points}
\end{table}

Based on this particular finding, we can assume that in the following analysis - the best vendors and products - the product preferences will be highly in favor of the beer beverages.
And thus the best vendor will probably be the organizer, as they sold all the beer beverages at the festival.

\begin{blue-box}{Key Takeaways}
	\begin{itemize}
		\item The best sale point was the \textbf{L20 PIVNÍ STAN 1} with total~\bfmtnum{10114}~orders processed, which was~\bfmtnump[2]{9.12}\% of the total orders created.
		\item The best sale point peaked at~\bfmtnum{840}~orders per hour.
		\item The \textbf{L20 PIVNÍ STAN 1} processed~\bfmtnum{9159}~unique customers, which was~\bfmtnump[2]{91.50}\% of all active customers.
	\end{itemize}
\end{blue-box}

In conclusion to these two questions, the results show clearly the busiest points of the festival and their ability to handle the load.
However, the results can be visualized in a more interactive way, which would provide a better understanding of the data.

One especially interesting visualization of the best sale and top-up points would be a heatmap of the festival area with the points and their respective transaction counts.
As this was initially intended to be part of the analysis, it was unfortunately not possible to create it due to the lack of the necessary data.

\subsubsection{Best Vendors}
\label{subsubsec:analysis-best-vendors}

To analyze the best vendors, in terms of performance, the same approach as with the sale points will be used.
\begin{gray-box}{Research Question}
	\textit{\researchq{performance-best-vendors}}
\end{gray-box}

The results in the~\autoref{tab:best-vendors} below show the distribution of the processed orders over the vendors.

\begin{table}[htbp]
	\centering
	\small
	\begin{tabularx}{\textwidth}{
		|>{\columncolor{unicorn_blue!5}\centering\arraybackslash}p{1cm}
		|>{\columncolor{unicorn_blue!5}\raggedright\arraybackslash}X
		|>{\columncolor{unicorn_blue!5}\raggedleft\arraybackslash}p{2.5cm}
		|>{\columncolor{unicorn_blue!5}\raggedleft\arraybackslash}p{2.5cm}|}
		\hline
		\rowcolor{unicorn_blue}
		\textbf{}
		& \textbf{\color{white}Vendor}
		& \textbf{\color{white}Customers}
		& \textbf{\color{white}Orders}
		\\\hline\hline
		% first rows
		\csvreader[
		head to column names,
		late after line={\\\hline},
		filter={\thecsvinputline<9}
		]{\ResultsDir/rq10-best-vendors.csv}{
			legal_name=\colentity,
			customer_count=\colcustomers,
			transaction_count=\coltrxcount,
		}{
			\the\numexpr\thecsvinputline-1
			& \colentity
			& \num[group-separator={,}]{\colcustomers}
			& \num[group-separator={,}]{\coltrxcount}
		}
		% separator
		\noalign{\vspace{1mm}}
		\multicolumn{5}{c}{\footnotesize{\textellipsis}}
		\\
		\noalign{\vspace{1mm}}
		\hline
		% last rows
		\csvreader[
		head to column names,
		late after line={\\\hline},
		filter={\thecsvinputline>25}
		]{\ResultsDir/rq10-best-vendors.csv}{
			legal_name=\colentity,
			customer_count=\colcustomers,
			transaction_count=\coltrxcount,
		}{
			\the\numexpr\thecsvinputline-1
			& \colentity
			& \num[group-separator={,}]{\colcustomers}
			& \num[group-separator={,}]{\coltrxcount}
		}
	\end{tabularx}
	\caption{Best Vendors}
	\label{tab:best-vendors}
\end{table}

As predicted in the previous section, the best vendor was the organizer, which processed the most orders and customers.
The total of~\bfmtnum{89217}~orders was processed by the organizer, which was~\bfmtnump[2]{80.04}\% of the total orders created and served~\bfmtnum{9831}~unique customers, which was~\bfmtnump[2]{98.22}\% of all active customers.

The second-best vendor, out of~\bfmtnum{27}~total, was some \textbf{Seller 05} with about~\bfmtnum{84104} orders less than the organizer.

In these results, we did not go after the hourly maximums, as in the previous questions, as the vendors were not time-bound and could have been at multiple places simultaneously.
The results would not be so relevant and would not provide any additional insights.

\begin{blue-box}{Key Takeaways}
	\begin{itemize}
		\item The best vendor was the organizer, which processed~\bfmtnum{89217}~orders (\bfmtnump[2]{80.04}\%~of the total orders).
		\item The best vendor served~\bfmtnum{9831}~unique customers (\bfmtnump[2]{98.22}\%~of all active customers).
		\item The second-best vendor was behind around~\bfmtnum{84104}~orders less than the best vendor.
	\end{itemize}
\end{blue-box}

\subsubsection{Best Products}
\label{subsubsec:analysis-best-products}

The last focus will be on the best products, which are the products that were sold the most during the event.

\begin{gray-box}{Research Question}
	\textit{\researchq{performance-best-products}}
\end{gray-box}

Previously, the best products were predicted to be the beer beverages, as the organizer sold all the beer beverages at the festival.

The results in the~\autoref{tab:best-products} below somehow confirm this prediction as the very best product was a depositable cup – \textbf{Kelímek – záloha} and the next four best products were beer beverages:

\begin{table}[htbp]
	\centering
	\small
	\begin{tabularx}{\textwidth}{
		|>{\columncolor{unicorn_blue!5}\centering\arraybackslash}p{1cm}
		|>{\columncolor{unicorn_blue!5}\raggedright\arraybackslash}X
		|>{\columncolor{unicorn_blue!5}\raggedleft\arraybackslash}p{2.5cm}
		|>{\columncolor{unicorn_blue!5}\raggedleft\arraybackslash}p{2.5cm}
		|>{\columncolor{unicorn_blue!5}\raggedleft\arraybackslash}p{2.5cm}|}
		\hline
		\rowcolor{unicorn_blue}
		\textbf{}
		& \textbf{\color{white}Product}
		& \textbf{\color{white}Sales}
		& \textbf{\color{white}Refunds}
		& \textbf{\color{white}Customers}
		\\\hline\hline
		% first rows
		\csvreader[
		head to column names,
		late after line={\\\hline},
		filter={\thecsvinputline<9}
		]{\ResultsDir/rq11-best-products.csv}{
			product_name=\colproduct,
			customer_count=\colcustomers,
			sales_count=\colsalescount,
			refund_count=\colrefundcount
		}{
			\the\numexpr\thecsvinputline-1
			& \colproduct
			& \num[group-separator={,}]{\colsalescount}
			& \num[group-separator={,}]{\colrefundcount}
			& \num[group-separator={,}]{\colcustomers}
		}
		% separator
		\noalign{\vspace{1mm}}
		\multicolumn{5}{c}{\footnotesize{\textellipsis}}
		\\
		\noalign{\vspace{1mm}}
		\hline
		% last rows
		\csvreader[
		head to column names,
		late after line={\\\hline},
		filter={\thecsvinputline>326}
		]{\ResultsDir/rq11-best-products.csv}{
			product_name=\colproduct,
			customer_count=\colcustomers,
			sales_count=\colsalescount,
			refund_count=\colrefundcount
		}{
			\the\numexpr\thecsvinputline-1
			& \colproduct
			& \num[group-separator={,}]{\colsalescount}
			& \num[group-separator={,}]{\colrefundcount}
			& \num[group-separator={,}]{\colcustomers}
		}
	\end{tabularx}
	\caption{Best Products}
	\label{tab:best-products}
\end{table}

As there were more than \bfmtnum{300}~products sold during the event, a better visualization of the results would be a bar chart showing the best product categories instead of individual products.
This can be seen in the~\autoref{fig:best-product-category} below.

\begin{figure}[H]
	\centering
	\includegraphics[width=0.99\textwidth]{\ResultsDir/rq11-best-product-category}
	\caption{Best Products by Category}
	\label{fig:best-product-category}
\end{figure}
\todo{Better chart} % TODO

This chart now confirms the prediction about the beer beverages, as the \textbf{Beer} (Pivo) category was the most sold during the event with a little more than~\bfmtnum{40000}~orders processed.

\begin{blue-box}{Key Takeaways}
	\begin{itemize}
		\item The best product was a depositable cup followed by beer beverages.
		\item Prediction about the beer beverages was confirmed, as the \textbf{Beer} (Pivo) category was the most sold during the event.
	\end{itemize}
\end{blue-box}

This last analysis provided insights into the best products, which confirmed the previous predictions and will now serve as a basis for the next section, where the beverage consumption will be analyzed.

\subsection{Summary}
\label{subsec:analysis-performance-indicators-summary}

Thanks to this section, the performance indicators of the event were analyzed, and the key metrics were identified.
The results analyzed the transactional processing performance, identified several peak points during the event, and provided insights into the best sale points, top-up points, vendors, and products.
Providing a better understanding of the event's performance and giving more context for the next analysis dealing with the beverage consumption.


\section{Beverage Consumption Analysis}
\label{sec:analysis-beverage-consumption}
\todo{-}


\section{Customer Analysis}
\label{sec:analysis-customers}
\todo{-}


	% chapter - implementation of the analytical dashboard
	%%% Dashboard Implementation
%%%%%%% Wording: ⏳
%%%%%%% Styling: ⏳
%%%%%%% References: ⏳
%%%%% Grammar: ⏳
%%% --------------------------------------------------------------
\chapter{Dashboard Implementation}
\label{ch:dashboard-implementation}

\todo{This chapter describes the implementation of the analytical dashboard using Dash and Plotly.}

\todo{Tools and technologies used - Dash, Plotly, Python, PostgreSQL, SQLAlchemy, Pandas, Mantine Dash Components etc.}
\todo{Describe async loops for data fetching and Dash background callbacks, f*cking issues with diskcache & DiskcacheManager issues with SQLIte concurrent connections, dscribe how we tried for 12hours straight to fix it and switch to celery + redis for async tasks but then switched back to diskcache with one-liner fix, fml}
\todo{Describe DB query management - QueryDefinition, QueryRegistry, QueryManger, QueryParameter and other internal classes}
\todo{Describe additional simple FS query caching mechanism}
\todo{Describe process of implementing individual dashboard sections, structure and most importantly the chosen visualizations/presentation of data}
\todo{Defend why we didn't go for deployment to cloud – sorry not sorry I'm not a lunatic, hate Python env and deployment issues}
\todo{Final dashboard presentation and features}
\todo{Next steps and missing features and future improvements we would do if we had more time}
	% chapter - conclusion
	%%% Conclusion
%%%%%%% Wording: ⏳
%%%%%%% Styling: ⏳
%%%%%%% References: ⏳
%%%%% Grammar: ⏳
%%% --------------------------------------------------------------
\chapter{Conclusion}
\label{ch:conclusion}


\section{Summary of Work}
\label{sec:conclusion-summary}
\todo{Write a summary of the work done in this thesis, recap the main objectives and how they were achieved. Note the key contributions of the research and practical side.}


\section{Reflections}
\label{sec:conclusion-reflections}
\todo{Write reflections on the project, what was learned, how the project could change festival operations, etc.}


\begin{section}{Professional Impact and Outcomes}
	\label{sec:future-impact}

	The main personal motivation for this thesis was to learn more about the data and demonstrate the findings in an interactive way, leading to more insights and knowledge for future updates of the NFCtron Hub dashboard.

	The research and analysis process has already given new valuable insights and knowledge about the data, the SQL techniques and optimization methods.
	Thanks to this, I was able to manage and deliver, together with my team, two significant updates to NFCtron Hub's analytical capabilities in late 2024.

	These included new time-based visualizations for ticket sales breakdowns and a comprehensive timeline view of festival activities including sales, refunds, arrivals, and customer ratings.
	A simple preview of this new timeline component can be seen in~\autoref{fig:hub-chart-update} below.

	\begin{figure}[H]
		\centering
		\includegraphics[width=\textwidth]{\ThesisFigures/ui/hub-chart-update}
		\caption{New Timeline Analysis in NFCtron Hub (December 2024 Update)}
		\label{fig:hub-chart-update}
	\end{figure}

	Also, thanks to the analytical findings and problems encountered during the analysis, I was able to provide valuable feedback to the development team.
	This will lead to improvements in the data collection process and the data model of the system, which will make future analysis easier and more efficient.

	Finally, the knowledge acquired by this thesis still shapes our product development and enables us to provide more advanced analytical capabilities to our clients – festival organizers.
	Future development will build upon these foundations, further expanding the analytical capabilities of NFCtron's system.
\end{section}
%%% bibliography
	\begin{flushleft}
		\bibliography{/Users/filipditrich/University/master_thesis/thesis/main,/Users/filipditrich/University/master_thesis/thesis/auto}
	\end{flushleft}
%%% list of figures
	\listoffigures
	\textit{Note: Unless stated otherwise, all figures are the author's own work.}
%%% list of tables
	\listoftables
%%% list of source codes
	\listoflistings
%%% list of abbreviations / acronyms
	%%% List of Acronyms
%%%%%%% Wording: ⏳
%%%%%%% Styling: ⏳
%%%%%%% References: ⏳
%%%%% Grammar: ⏳
%%% --------------------------------------------------------------
\chapter*{List of Acronyms}
\addcontentsline{toc}{chapter}{List of Acronyms}
\printacronyms[
    name=List of acronyms,
    heading=none,
    sort=true,
    list-style=longtable,
    extra-style=plain
]

%%% appendix
	%%% List of Appendices
%%%%%%% Wording: ✅
%%%%%%% Styling: ✅
%%%%%%% References: ✅
%%%%% Grammar: ✅
%%% --------------------------------------------------------------
\appendix
\addtocontents{toc}{\protect\setlength{\cftsecnumwidth}{27mm}}
\chapter*{List of Appendices}
\addcontentsline{toc}{chapter}{List of Appendices}
\renewcommand{\thesection}{Appendix \Alph{section}}

%%% Appendix A: Source Code of the Dashboard
%%%%% File preparation: ✅
%%%%% Instructions: ✅
%%% --------------------------------------------------------------
\begin{section}{Source Code of the Dashboard}
	\label{appendix:source-code}
	The source code for the dashboard application is available on the attached CD in the~\texttt{dashboard\_app.zip} file,
	alongside the Thesis itself and Extended Summary in PDF a format\footnote{Both the source code and thesis are also available on a~\href{https://github.com/filipditrich/master-thesis}{\texttt{GitHub repository}}}.

	The application is built using Python 3.9\footnote{\url{https://www.python.org/downloads/release/python-390/}}, and its requirements are listed in the~\texttt{requirements.txt} file.
	However, as explained in the thesis, this application depends on a local PostgreSQL database, which is not included in the source code.

	The application is structured as follows:
	\begin{itemize}
		\item \texttt{assets/}: directory with CSS stylesheets
		\item \texttt{sections/}: directory with implementation of dashboard sections
		\item \texttt{\_chart\_utils.py}: utility functions for chart generation, SankeyDiagram implementation
		\item \texttt{\_dash\_utils.py}: utility functions for Dash
		\item \texttt{\_db\_utils.py}: common database utility functions with query manager implementation
		\item \texttt{\_format\_utils.py}: utility functions for data formatting
		\item \texttt{\_query\_manager.py}: registered SQL queries
		\item \texttt{app.py}: the main Dash application implementation
	\end{itemize}

\end{section}
%%% extended summary
	\addtocontents{toc}{\protect\contentsline{chapter}{Extended Summary}{}{}}
\end{document}

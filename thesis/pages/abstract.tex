%%% Abstract
%%%%%%% Wording: ⏳
%%%%%%% Styling: ⏳
%%%%%%% References: ⏳
%%%%% Grammar: ⏳
%%% --------------------------------------------------------------
\newpage
\pagestyle{plain}

%%% abstract in english
\nobreak\vbox to 0.49\vsize{
	\setlength\parindent{0mm}
	\setlength\parskip{5mm}

	%%% title
	{\large\bfseries Abstract}

	%%% text
	\noindent
	This thesis analyzes payment transaction data from a larger Czech festival that utilized the NFCtron payment system.
	The research is conducted on over \fmtnum{141000} transactions from more than \fmtnum{10000} unique attendees and focuses on 29 research questions related to Cashflow and Revenue Sources, System Performance, Beverage Consumption, and Customer Behavior analysis.

	The methodology ranges from establishing a local environment for the analysis, through data extraction cleaning and preparation for local database, to data anonymization due to privacy concerns.

	All 29 research questions are thoroughly answered and presented with rich visuals using a variety of charts and tables in each of the four major research areas.
	Moreover, the findings led to the development of an interactive analytical dashboard prototype using Dash and Plotly technologies, that demonstrates the key insights.

	Process of the analysis and dashboard development is thoroughly described in the thesis, including the main technical challenges faced and used solutions to overcome them.
	Although the dashboard app is still in the prototype stage, the analytical findings, data obtaining techniques and anonymization methods provide a solid foundation for future development and have already contributed to improvements in personal line of work.

	\textit{Keywords: festival data analysis, cashless payments, data visualization, data anonymization, interactive analytic dashboard, Python, Dash and Plotly}
	\vss}

%%% abstract in czech
%\vbox to 0.5\vsize{
%	\setlength\parindent{0mm}
%	\setlength\parskip{5mm}
%
%	%%% title
%	{\large\bfseries Abstrakt}
%
%	%%% text
%	\noindent
%	Tato práce analyzuje transakční data z většího českého festivalu, který využíval platební systém NFCtron, a zaměřuje se na 29 výzkumných otázek týkajících se pěněžních toků, výkonu systému, konzumace nápojů a chování zákazníků.
%	Analýza zahrnuje přes~\fmtnum{141000} transakcí od více než~\fmtnum{10000} unikátních návštěvníků, poskytující komplexní pohled na dynamiku festivalu.
%
%	Metodologie zahrnovala vytvoření lokálního analytického prostředí, implementaci technik anonymizace dat, které zajistily ochranu soukromí a zároveň zachovaly analytickou hodnotu, až po zodpovězení všech výzkumných otázek.
%	Klíčové poznatky o dynamice festivalu byly odhaleny a vedly k vývoji prototypu interaktivního analytického dashboardu, který demonstruje jejich výsledky.
%
%	Ačkoliv je dashboard stále ve fázi prototypu, analytická architektura a zjištění poskytují základ pro budoucí vývoj a již přispěly k zlepšení analytických schopností NFCtron.
%
%	\textit{Klíčová slova: analýza dat z festivalu, bezhotovostní platby, vizualizace dat, anonymizace dat, interaktivní analytický dashboard, Python, Dash a Plotly}
%	\vss}

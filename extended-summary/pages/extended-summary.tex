%%% Chapter - Extended Summary
%%%%% Wording: ⏳
%%%%% Styling: ⏳
%%%%% References: ⏳
%%%%% Grammar: ⏳
%%% --------------------------------------------------------------
\chapter*{Extended Summary}
\label{ch:extended-summary}
Payments at festivals are an important part of successful event management.
The transition from cash to cashless payments has been a significant trend over the last decade, with many benefits for both festival organizers and attendees.

Traditional cashless payment systems based on payment terminals, face two major challenges: they are expensive to implement reliably in venues with unstable internet connections, and they provide only limited insights into the transaction data they generate.

NFCtron addresses these challenges by providing a comprehensive solution that facilitates, other than standard card terminal payments, also a NFC-based chip bracelets payment system, both with a powerful B2B analytics platform.
The system offers end-to-end functionality, from initial online ticket sales to on-site operations and post-event settlement, allowing organizers to concentrate on core event aspects such as marketing and customer experience.

Founded in 2019, NFCtron is a Czech company that has evolved from surviving the COVID-19 epidemic to being the leading cashless payment provider for events and festivals in the Czech Republic and is currently expanding to other CE countries.
The company's success mostly comes from its focus on giving event organizers useful data insights so they may maximize their events and make better data-driven decisions.

Though NFCtron Hub, which offers real-time analytics and key performance indicators, has a lot of data available, there is still great unrealized potential in the data.
Although the platform offers basic statistics including total sales, refunds, and chip balances as well as several data exports, organizers—who are usually not data scientists—need simpler ways for gaining more in-depth understanding from this data.

%%% Section - Goals and Objectives
%%% --------------------------------------------------------------
\section*{Goals and Objectives}
\label{sec:goals-and-objectives}
Focusing on a significant Czech festival that adopted the NFCtron system in 2024, this thesis aims to analyze the data generated by the system and provide insights that can be used to improve future events.
Working with a disclosed event organizer, the study answers \textbf{29 research questions} in four key domains:\\
\begin{itemize}
	\item \textbf{Cashflow and Revenue Sources Analysis}: an examination of the event's financial performance, including revenue composition, top-up patterns, and sales distribution
	\item \textbf{Performance Indicators Analysis}: the assessment of system effectiveness through operational metrics, peak volumes, and transaction processing
	\item \textbf{Beverage Consumption Analysis}: examining brand preferences, returnable cup usage, and consumption patterns
	\item \textbf{Customer Analysis}: comprehending customer segmentation, payment behaviors, and attendance patterns
\end{itemize}

The technical goals consist of analyzing data from over~\bfmtnum{141000} transactions generated by over~\bfmtnum{10000} attendees while ensuring data privacy through suitable anonymization.
Additionally, the study aims to develop an interactive dashboard prototype that visualizes the insights and provides a user-friendly interface for exploring the data.

The scope specifically excludes real-time monitoring and multiple event comparisons in order to focus on post-event analysis of a single festival.
Although the study uses NFCtron system data, it does not directly integrate with the current NFCtron Hub platform.
It rather focuses on developing a prototype dashboard to demonstrate the key findings and insights that were found thanks to the analysis.

Although the study acknowledges constraints such as anonymized data requirements and the single-event focus, it aims to provide a comprehensive analysis that can be applied to similar large-scale events.
The results are intended to be actionable and valuable for festival organizers, while the dashboard prototype aims to demonstrate the potential of advanced data analytics in event management.

%%% Section - Employed Methods
%%% --------------------------------------------------------------
\section*{Employed Methods}
\label{sec:employed-methods}
To achieve its stated objectives, this thesis used a variety of data analysis methods and dashboard development approaches.

The analysis process began with data exploration and cleaning, which was followed by a custom anonymization process that protected privacy while preserving analytical value.
Additional details, such as beverage volumes, were added to the dataset, which was then processed using prepared custom SQL functions.

The results from the analysis were visualized using a combination of charts, diagrams, and tables to provide clear and understandable insights, based on the best practices in data visualization.

The dashboard development process followed an iterative approach, reading documentation and tutorials to learn the necessary technologies and tools.
The prototype was built using Python's Dash framework, which allowed for interactive visualizations and dynamic data filtering.

\pagebreak[4]

%%% Section - Major Sources
%%% --------------------------------------------------------------
\section*{Major Sources}
\label{sec:major-sources}
Many materials have helped to shape and enhance this thesis.
While some have offered important viewpoints on payment systems and data handling, others have offered thorough technical insights into data processing and visualization.

These are the most crucial references and how they have helped to produce this work:

\subsection*{PostgreSQL Documentation}
\label{subsec:postgresql-documentation}
The comprehensive database documentation served as the foundation for our data handling approach, providing essential guidance for query optimization and data structure design in handling large volumes of festival transaction data.

\small{\fullcite{tpgdg_17_postgresql_17_a4_pdf}}

\subsection*{Dash Framework Documentation}
\label{subsec:dash-framework-documentation}
As the primary resource for dashboard development, the Dash documentation and Mantine Components guides were instrumental in implementing interactive data visualizations and creating an effective user interface.

\small{\fullcite{plotly_dash_plotly_com}}
\small{\fullcite{sv_snehilvj_dash_mantine_components}}

\subsection*{Data Anonymization Research}
\label{subsec:data-anonymization-research}
An interesting article on data anonymization techniques guided the core approach to protecting sensitive festival data while maintaining its analytical value.

\begin{flushleft}
	\small{\fullcite{hd_data_anonymization_techniques}}
\end{flushleft}

\subsection*{Festival Payment Systems Analysis}
\label{subsec:festival-payment-systems-analysis}
The analysis of cashless payment adoption in festivals provided essential context for understanding the broader impact and importance of digital payment systems in event management.

\small{\fullcite{bl_en_waarom_festivals_overstappen_op_cashless_betalen}}

\subsection*{Knowledge Discovery Process}
\label{subsec:knowledge-discovery-process}
The KDD framework helped structure our analytical methodology, ensuring a systematic approach to extracting meaningful insights from raw festival data.

\small{\fullcite{uord_kdd_1_kdd}}

\pagebreak[4]

%%%% Section - Summary
%%%% --------------------------------------------------------------
\section*{Summary}
\label{sec:summary}
This research focused on two key goals: analyzing NFCtron's festival payment data and developing an interactive dashboard to visualize the findings.
The analysis revealed significant insights about festival operations, including direct and indirect revenue streams, peak system performance of~\bfmtnum{8986}~transactions per hour,
and customer behavior patterns showing~\bfmtnump[2]{40.69}\% mobile app usage with significant iOS dominance.

Thanks to this thesis, it was already possible to deliver few improvements to the NFCtron system, including new analytics features for festival organizers.
While the interactive dashboard prototype successfully demonstrated key insights, time constraints unfortunately limited some planned features.

Along with the data, the thesis gave important information about how to analyze it and create interactive dashboards to visualize the results.
It demonstrated how important clear research questions, good data, processing, and efficient data presentation are for getting good analysis outcomes.
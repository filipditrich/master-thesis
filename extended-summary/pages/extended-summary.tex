%%% Chapter - Extended Summary
%%%%% Wording: ✅
%%%%% Styling: ✅
%%%%% References: ✅
%%%%% Grammar: ✅
%%% --------------------------------------------------------------
\chapter*{Extended Summary}
\label{ch:extended-summary}
Payments at festivals are an important part of successful event management.
The transition from cash to cashless payments has been a significant trend over the last decade, with many benefits for both festival organizers and attendees.

Traditional cashless payment systems based on payment terminals, face two major challenges: they are expensive to implement reliably in venues with unstable internet connections, and they provide only limited insights into the transaction data they generate.

NFCtron addresses these challenges by providing a comprehensive solution that facilitates, other than standard card terminal payments, also a NFC-based chip bracelets payment system, both with a powerful B2B analytics platform.
The system offers end-to-end functionality, from initial online ticket sales to on-site operations and post-event settlement, allowing organizers to concentrate on core event aspects such as marketing and customer experience.

Founded in 2019, NFCtron is a Czech company that has evolved from surviving the COVID-19 epidemic to being the leading cashless payment provider for events and festivals in the Czech Republic and is currently expanding to other CE countries.
The company's success mostly comes from its focus on giving event organizers useful data insights so they may maximize their events and make better data-driven decisions.

Though NFCtron Hub, which offers real-time analytics and key performance indicators, has a lot of data available, there is still great unrealized potential in the data.
Although the platform offers basic statistics including total sales, refunds, and chip balances as well as several data exports, organizers—who are usually not data scientists—need simpler ways for gaining more in-depth understanding from this data.

%%% Section - Goals and Objectives
%%% --------------------------------------------------------------
\section*{Goals and Objectives}
\label{sec:goals-and-objectives}
The main goal of this thesis is to analyze a disclosed Czech festival, which used the NFCtron cashless payment solution, answer a set of research questions, present the findings, and develop a prototype of an interactive dashboard to visualize the results.

Working with a disclosed event organizer, the analysis answers \textbf{29 research questions} in four key areas:\\
\begin{itemize}
	\item \textbf{Cashflow and Revenue Sources Analysis}: Analysis of the overall financial performance, including detecting various revenue sources, top-up patterns, and total sales distribution.
	\item \textbf{Performance Indicators Analysis}: Analyzes the NFCtron system's performance, including peak transaction rates, system usage, and possible bottlenecks.
	\item \textbf{Beverage Consumption Analysis}: Analysis of beverage brand preferences, returnable cup usage, and customer consumption patterns.
	\item \textbf{Customer Analysis}: Customer segmentation and behavioral analysis, focusing on identifying customer composition and their behavior patterns.
\end{itemize}

The main technical goal consists of thorough and ethical data analysis of over~\bfmtnum{141000} transactions from more than~\bfmtnum{10000} unique attendees, ensuring data privacy through suitable anonymization.
The results are then presented and an interactive dashboard prototype is developed to visualize the key insights found in the analysis and thus provide a more user-friendly and intuitive way to explore the data.

The scope specifically excludes real-time monitoring and multiple event comparisons to focus on the post-event analysis of a one single festival.

Although the analysis uses NFCtron system data, it does not directly integrate with the current NFCtron platform in any way.
It rather focuses on developing a prototype dashboard to demonstrate the key findings and insights that were found thanks to the analysis.

The intention of the analysis results is to be clear and valuable new insights for the festival organizer, while the dashboard prototype aims to demonstrate the potential of advanced data analytics leveraging the NFCtron data.

%%% Section - Employed Methods
%%% --------------------------------------------------------------
\section*{Employed Methods}
\label{sec:employed-methods}
Various methods were employed to achieve the objectives of this thesis, including data exploration, cleaning, and preparation, as well as data anonymization to protect privacy.

The analysis began with an initial data exploration, extraction and cleaning process, followed by a custom anonymization process to protect privacy while preserving the important analytical value.
To answer all stated research questions, additional data and modifications were made to the dataset and local database itself, including beverage volumes and custom SQL functions.

The results from the analysis were visualized using a combination of charts, diagrams, and tables to provide clear and understandable insights, based on the best practices in data visualization.

Dashboard development has required careful planning and an iterative approach, with most time spent on studying documentation and learning the necessary technologies and tools.
The prototype was built using a Python dashboard framework Dash, which allowed for interactive visualizations and dynamic data filtering.

\pagebreak[4]

%%% Section - Major Sources
%%% --------------------------------------------------------------
\section*{Major Sources}
\label{sec:major-sources}
Many materials from various sources, yet mostly online sources, have helped to realize this thesis.
Some of the most crucial references and how they have helped to produce this work are listed below.

\subsection*{PostgreSQL Documentation}
\label{subsec:postgresql-documentation}
Without much prior experience with PostgreSQL, the database documentation was crucial in setting up the local database and optimizing the data extraction process.

\small{\fullcite{tpgdg_17_postgresql_17_a4_pdf}}

\subsection*{Dash Framework Documentation}
\label{subsec:dash-framework-documentation}
As the primary resource for dashboard development, the Dash documentation and Mantine Components guides were instrumental in implementing interactive data visualizations and creating an effective user interface.

\small{\fullcite{plotly_dash_plotly_com}}
\small{\fullcite{sv_snehilvj_dash_mantine_components}}

\subsection*{Data Anonymization Research}
\label{subsec:data-anonymization-research}
An interesting article on data anonymization techniques guided the core approach to protecting sensitive festival data while maintaining its analytical value.

\begin{flushleft}
	\small{\fullcite{hd_data_anonymization_techniques}}
\end{flushleft}

\subsection*{Festival Payment Systems Analysis}
\label{subsec:festival-payment-systems-analysis}
The analysis of recent cashless payment systems adoption in festivals confirmed the importance of digital payment systems in event management and provided more context for understanding the broader impact of the NFCtron solution.

\small{\fullcite{bl_en_waarom_festivals_overstappen_op_cashless_betalen}}

\subsection*{Knowledge Discovery Process}
\label{subsec:knowledge-discovery-process}
The KDD framework helped structure the analytical methodology, helping with a more systematic approach to extracting meaningful information from the raw festival data.

\small{\fullcite{uord_kdd_1_kdd}}

\pagebreak[4]

%%%% Section - Summary
%%%% --------------------------------------------------------------
\section*{Summary}
\label{sec:summary}
This research focused on two key goals: analyzing NFCtron's festival payment data and developing an interactive dashboard to visualize the findings.
The analysis revealed significant insights about the festival performance, including direct and indirect revenue streams, peak system performance of~\bfmtnum{8986}~transactions per hour,
and customer behavior patterns showing~\bfmtnump[2]{40.69}\% mobile app usage with significant iOS dominance.

Thanks to this thesis, it was already possible to deliver various improvements to the NFCtron system, including new analytics features for festival organizers.
While the interactive dashboard prototype successfully demonstrated key insights, time constraints unfortunately limited some planned features.

Along with the data, the thesis gave important information about how to analyze it and create interactive dashboards to visualize the results.
It demonstrated how important clear research questions, good data, processing, and efficient data presentation are for getting good analysis outcomes.